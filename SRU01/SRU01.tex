\documentclass[12pt]{article}
\usepackage{anysize}
\usepackage[ngerman,english]{babel}
\marginsize{3.5cm}{2.5cm}{1cm}{2cm}

\usepackage[utf8]{inputenc}
\usepackage[english]{babel}
\usepackage{amsmath}
\usepackage{amsthm}
\usepackage{amsfonts}
\usepackage{scalerel,amssymb}  % for mathbb
%---------------------------------------------------------
\usepackage{authblk}
\usepackage{blindtext}
\usepackage{graphicx}
\usepackage[caption=true]{subfig}
\usepackage[
	colorlinks = true,
	citecolor = red,
	% linkcolor = darkblue, % internal references
	% urlcolor = darkblue,
]{hyperref}
\usepackage{cleveref}
\crefname{figure}{Figure}{Figures}
\crefname{equation}{Eq.}{Eqs.}
%---------------------------------------------------------
% \newcommand{\begin{equation}}{\eqbeg} 
% \newcommand{\end{equation}}{\eqend}
\usepackage{bm}
\usepackage{dfcmd}
% \newcommand{\Bvarepsilon}{\bm\varepsilon} 
% \newcommand{\Bx}{\bm x} 
% \newcommand{\Bu}{\bm u} 
% \newcommand{\Bv}{\bm v} 
% \newcommand{\Bg}{\bm g} 
% \newcommand{\Bb}{\bm b} 
\newcommand{\divv}{\text{divv}}
%---------------------------------------------------------
\usepackage{lineno}
% \linenumbers     

\usepackage{float}
\usepackage{psfrag}

\usepackage{fontawesome}
\usepackage{relsize}

%---------------------------------------------------------
%For \toprule \midrule \bottomrule in table environment
\usepackage{booktabs}

%For theorem into a box
\usepackage{mdframed}
% \newmdtheoremenv{theo}{Theorem}
\newmdtheoremenv{interestingfactboxed}{Interesting fact}
\newmdtheoremenv{observationboxed}{Observation}
\newmdtheoremenv{exampleboxed}{Example}
\newmdtheoremenv{questionboxed}{Question}
\newmdtheoremenv{summaryboxed}{Summary}

%For big integral sign
\usepackage{bigints}

%For [i)] in enumerate
\usepackage{enumerate}

%For cancelto
\usepackage{cancel}
%Page numbering in style 1/3...
\usepackage{lastpage}  
\usepackage{hyperref}
\makeatletter
\renewcommand{\@oddfoot}{\hfil 
% Aachen, November $04^{th}$, 2021 \hspace{300pt} 
Mathe III $\cdot$ SRU00 $\cdot$ WS22/23 
\hspace{280pt} 
\thepage/\pageref{LastPage}\hfil}
\makeatother
%------------------------------------------------------------------------------

\begin{document}
\begin{center}
	\section*{Self-exercise - SRU01 - Response to questions}
\end{center}
\begin{center}
	Tuan Vo
\end{center}
\begin{center}
	$23^{\text{th}}$ October, 2022
\end{center}
%------------------------------------------------------------------------------
\section{Step-by-step with A1a-SRU01}
\begin{exampleboxed}
	Examine the first variation of the functional $F$ in $v$ direction
	\begin{align}
		\label{eq:functionalexample}
		F:
		\begin{cases}
			D \rightarrow \mathbb{R}, \\
			\displaystyle
			y \mapsto
			F(y):= 
			\frac{1}{2}
			\bigintsss_{-1}^1
			\left( x\,y'(x) \right)^2 dx,
		\end{cases}
	\end{align}
	where 
	\begin{align}
		y \in D     & = \left\{ \omega \in C^2([-1,1]): \omega(-1)=\omega(1)=1 \right\}, 
		\label{eq:conditiony}
		\\
		v \in D_{0} & = \left\{ \omega \in C^2([-1,1]): \omega(-1)=\omega(1)=0 \right\}.
		\label{eq:conditionv}
	\end{align}
\end{exampleboxed}

Approach: By using the definition, as follows 
\begin{align}
	\label{eq:variationformular}
	\boxed{
		\delta F(y;v) := \lim_{t\to 0}\frac{F(y+tv)-F(y)}{t}.
	}
\end{align}

In order to find the variation of the functional $F$,
which fulfills 
\eqref{eq:functionalexample}
and 
\eqref{eq:conditiony},
in $v$ direction satisfying
\eqref{eq:conditiony}, 
we need to compute the limit, by definiton,
as shown in \eqref{eq:variationformular}.
First, we shall write down the corresponding term 
for $F(y+t\,v)$ based on
the functional $F(y)$ given in
\eqref{eq:functionalexample}, which means
\begin{align}
	\label{eq:termextra}
	F(y+t\,v)
	= 
	\bigintsss_{-1}^{1}
	\Big(
	x \,
	(y+t\,v)^\prime(x)
	\Big)^2\,dx,
\end{align}
then by substituting both \eqref{eq:termextra}
and \eqref{eq:functionalexample}
into \eqref{eq:variationformular}
we obtain 
\begin{align}
	\delta F(y;v)
	 & :=  \lim_{t\to 0}\frac{F(y+t\,v)-F(y)}{t} \notag \\
	 & =  \lim_{t\to 0} \frac{1}{t}
	\left(
	\frac{1}{2}
	\bigintsss_{-1}^{1}
	\Big(
	x \,
	(y+t\,v)^\prime(x)
	\Big)^2\,dx
	-
	\frac{1}{2}
	\bigintsss_{-1}^{1}
	\Big(
	x \, y^\prime(x)
	\Big)^2\,dx
	\right)\notag                                       \\
	 & =  \lim_{t\to 0} \frac{1}{2t}
	\left(
	\bigintsss_{-1}^{1}
	\Big(
	x \,
	(y(x)+t\,v(x))^\prime
	\Big)^2\,dx
	-
	\bigintsss_{-1}^{1}
	\Big(
	x \, y^\prime(x)
	\Big)^2\,dx
	\right), 
\end{align}
where we will now compute those terms within the parenthesis, as follows
\begin{align}
	\delta F(y;v)
	 & =  \lim_{t\to 0} \frac{1}{2t}
	\left(
	\bigintsss_{-1}^{1}
	\Big(
	x \,
	(y^\prime(x)+t\,v^\prime(x))
	\Big)^2\,dx
	-
	\bigintsss_{-1}^{1}
	\Big(
	x \, y^\prime(x)
	\Big)^2\,dx
	\right)\notag                    \\
	 & =  \lim_{t\to 0} \frac{1}{2t}
	\left(
	\bigintsss_{-1}^{1}
	x^2 \,
	\Big(y^\prime(x)+t\,v^\prime(x)\Big)^2
	\,dx
	-
	\bigintsss_{-1}^{1}
	x^2 \, (y^\prime(x))^2\,dx
	\right)\notag                    \\
	 & =  \lim_{t\to 0} \frac{1}{2t}
	\!
	\bigintsss_{-1}^{1}
	\!\!\!\!
	x^2 \,
	\Big(
	(y^\prime(x))^2 
	+ 2\,y^\prime(x)\,t\,v^\prime(x)
	+ t^2\,(v^\prime(x))^2
	\Big)
	\,dx
	-\!\!\!
	\bigintsss_{-1}^{1}
	\!\!\!\!
	x^2 \, (y^\prime(x))^2\,dx\notag \\
	 & =  \lim_{t\to 0} \frac{1}{2t}
	\bigintsss_{-1}^{1}
	\!\!\!\!
	\Big(
	\cancel{x^2\,(y^\prime(x))^2}
	+ 2\,x^2\,y^\prime(x)\,t\,v^\prime(x)
	+ x^2\,t^2\,(v^\prime(x))^2
	- \cancel{x^2\,(y^\prime(x))^2}
	\Big)
	\,dx
	\notag                           \\
	 & =  \lim_{t\to 0} \frac{1}{2t}
	\bigintsss_{-1}^{1}
	\Big(
	2\,x^2\,y^\prime(x)\,t\,v^\prime(x)
	+ x^2\,t^2\,(v^\prime(x))^2
	\Big)
	\,dx.
	\label{eq:reducedform}
\end{align}

Next, the expression in \eqref{eq:reducedform} is computed further,
by first computing terms within the integrand going with $t$
and then splitting into two terms going with the limit, as follows
\begin{align}
	\delta F(y;v)
	 & \stackrel{\eqref{eq:reducedform}}{=}
	\lim_{t\to 0} \frac{1}{2t}
	\bigintsss_{-1}^{1}
	\Big(
	2\,x^2\,y^\prime(x)\,t\,v^\prime(x)
	+ x^2\,t^2\,(v^\prime(x))^2
	\Big)
	\,dx\notag                              \\
	 & =
	\lim_{t\to 0}
	\bigintsss_{-1}^{1}
	\frac{2\,x^2\,y^\prime(x)\,t\,v^\prime(x)
		+ x^2\,t^2\,(v^\prime(x))^2}{2t}
	\,dx\notag                              \\
	 & =
	\lim_{t\to 0}
	\bigintsss_{-1}^{1}
	\Big(
	x^2\,y^\prime(x)\,v^\prime(x)
	+ \frac{1}{2}x^2\,t\,(v^\prime(x))^2
	\Big)
	\,dx\notag                              \\
	 & =
	\lim_{t\to 0}
	\bigintsss_{-1}^{1}
	x^2\,y^\prime(x)\,v^\prime(x)
	\,dx
	+ 
	\lim_{t\to 0}
	\frac{1}{2}
	\bigintsss_{-1}^{1}
	x^2\,t\,(v^\prime(x))^2
	\,dx \notag                             \\
	 & =
	\bigintsss_{-1}^{1}
	x^2\,y^\prime(x)\,v^\prime(x)
	\,dx
	+
	\underbrace{\lim_{t\to 0}
		\frac{t}{2}
		\bigintsss_{-1}^{1}
		x^2\,(v^\prime(x))^2
		\,dx}_{=0}.
	\label{eq:reducedform2}
\end{align}
Note in passing that the term $t$ can be brought out of the integral 
arising in \eqref{eq:reducedform2}
because $t$ is just a constant for this integral. 
It is also worthy to note that bringing a limit from outside 
to inside an integral is critical as the final result is not always the same.
Herein, for the limit showing in \eqref{eq:reducedform2}
we have been not doing so, but rather bringing the $t$ out of the integral.
Finally, the first variation of functional $F$ is processed further as follows
\begin{align}
	\label{eq:reducedform3}
	\delta F(y;v)
	\stackrel{\eqref{eq:reducedform2}}{=}
	\bigintsss_{-1}^{1}
	x^2\,y^\prime(x)\,v^\prime(x)
	\,dx.
\end{align}
We will compute \eqref{eq:reducedform3}
by taking into consideration the \textbf{integration by parts}, as follows 
\begin{align}
	\label{eq:integrationbyparts}
	\boxed{
		\bigintsss a'b =  ab - \bigintsss ab'
	}
\end{align}
where we will assign
\begin{align}
	\label{eq:assignedab}
	\begin{cases}
		a := v(x), \\
		b := x^2\,y'(x),
	\end{cases}
\end{align}
which leads to 
\begin{align}
	\delta F(y;v)
	 & \stackrel{\eqref{eq:reducedform3}}{=}
	\bigintsss_{-1}^{1}
	\underbrace{v^\prime(x)}_{a'}
	\underbrace{x^2\,y^\prime(x)}_{b}
	\,dx\notag                               \\
	 & 
	\stackrel{
		\stackrel{\mbox{\scriptsize$\eqref{eq:assignedab}$}}{\eqref{eq:integrationbyparts}}
	}{=}
	\Big(
	\underbrace{v(x)}_{a}
	\,
	\underbrace{x^2\,y'(x)}_{b}
	\Big)\Bigg|_{x=-1}^{x=1}
	-
	\bigintsss_{-1}^{1}
	\underbrace{v(x)}_{a}
	\underbrace{
		\frac{d(x^2\,y'(x))}{dx}
	}_{b'}\,dx\notag                         \\
	 & =
	\underbrace{v(1)}_{=0}\,
	(1)^2\,
	\underbrace{y'(1)}_{=1}
	- 
	\underbrace{v(-1)}_{=0}\,
	(-1)^2\,
	\underbrace{y'(-1)}_{=1}
	-\bigintsss_{-1}^{1}v(x)\frac{d(x^2\,y'(x))}{dx}\,dx
	\label{eq:somethingtonote}               \\
	 & =
	-\bigintsss_{-1}^{1}v(x)\frac{d(x^2\,y'(x))}{dx}\,dx.
\end{align}

Note in passing that the definition 
in \eqref{eq:conditiony}
and \eqref{eq:conditionv}
yield the following equalities
\begin{align}
	y(1)=y(-1)=1, \\
	v(1)=v(-1)=0,
\end{align}
which we have used in \eqref{eq:somethingtonote}.
Therefore, we obtain the final result 
\begin{align}
	\therefore\quad\boxed{
		\delta F(y;v)
		=
		-\bigintsss_{-1}^{1}v(x)\frac{d(x^2\,y'(x))}{dx}\,dx,
		\quad
		\forall\, v(x)\in D_{0},\, y(x)\in D. 
	}
\end{align}
% Let us now forget the argument $(x)$
% in \eqref{eq:first1} and \eqref{eq:first2}
% for a while, i.e. by dropping it temporarily, 
% just for a sake of a better overview. 
% Then, the new expression without $(x)$
% will take the form
% \begin{align}
% 	\delta F(y;v)
% 	 & :=  \lim_{t\to 0}\frac{F(y+s\,v)-F(y)}{t} \notag \\
% 	 & =  \lim_{t\to 0} \frac{1}{t}
% 	\bigintssss_{-1}^{1}
% 	\Big(
% 	(y+s\,v) \left(y+s\,v\right)^\prime
% 	+ \exp{\Big((y+s\,v)^{\prime 2}\Big)}
% 	\Big)\,dx \notag                                    \\
% 	% 
% 	 & \qquad\qquad\qquad\qquad\qquad\qquad
% 	-\lim_{t\to 0} \frac{1}{t}
% 	\bigintssss_{-1}^{1}
% 	\Big(
% 	y\,y' + \exp{\Big(y^{\prime 2}\Big)}
% 	\Big)\,dx                            
% 	\label{eq:first22}
% \end{align}
% which leads to further analysis, as follows
% \begin{align}
% 	\delta F(y;v)
% 	 & \stackrel{\eqref{eq:first22}}{:=}
% 	\lim_{t\to 0}\frac{F(y+s\,v)-F(y)}{t}   \notag \\
% 	%  & =  \lim_{t\to 0} \frac{1}{t}
% 	% \bigintssss_{-1}^{1}
% 	% \Big(
% 	% (y+s\,v) \left(y+s\,v\right)^\prime
% 	% + \exp{\Big((y+s\,v)^{\prime 2}\Big)}
% 	% \Big)\,dx
% 	% \label{eq:first123}                          \\
% 	% % 
% 	%  & \qquad\qquad\qquad\qquad\qquad\qquad
% 	% -\lim_{t\to 0} \frac{1}{t}
% 	% \bigintssss_{-1}^{1}
% 	% \Big(
% 	% y\,y' + \exp{\Big(y^{\prime 2}\Big)}
% 	% \Big)\,dx                            
% 	% \label{eq:first223}                          \\
% 	%-----------------------------------------------------
% 	 & =  \lim_{t\to 0} \frac{1}{t}
% 	\bigintssss_{-1}^{1}
% 	\Big(
% 	(y+s\,v) (y^\prime+s\,v^\prime)
% 	+ \exp{\Big((y+s\,v)^{\prime 2}\Big)}
% 	\Big)\,dx                               \notag \\
% 	% 
% 	 & \qquad\qquad\qquad\qquad\qquad\qquad
% 	-\lim_{t\to 0} \frac{1}{t}
% 	\bigintssss_{-1}^{1}
% 	\Big(
% 	y\,y' + \exp{\Big(y^{\prime 2}\Big)}
% 	\Big)\,dx                               \notag \\     
% 	%-----------------------------------------------------   
% 	 & =  \lim_{t\to 0} \frac{1}{t}
% 	\bigintssss_{-1}^{1}
% 	\Big(
% 	y\,y^\prime 
% 	+ s\,y\,v^\prime  
% 	+ s\,v\,y^\prime
% 	+ s^2\,v\,v^\prime
% 	+ \exp{\Big((y+s\,v)^{\prime 2}\Big)}
% 	\Big)\,dx                               \notag \\
% 	% 
% 	 & \qquad\qquad\qquad\qquad\qquad\qquad
% 	-\lim_{t\to 0} \frac{1}{t}
% 	\bigintssss_{-1}^{1}
% 	\Big(
% 	y\,y' + \exp{\Big(y^{\prime 2}\Big)}
% 	\Big)\,dx                               \notag \\ 
% 	%-----------------------------------------------------   
% 	 & =  \lim_{t\to 0} \frac{1}{t}
% 	\bigintssss_{-1}^{1}
% 	\Big(
% 	y\,y^\prime 
% 	+ s\,y\,v^\prime  
% 	+ s\,v\,y^\prime
% 	+ s^2\,v\,v^\prime
% 	- y\,y^\prime
% 	\Big)\,dx                               \notag \\
% 	% 
% 	 & \qquad\qquad\qquad\qquad
% 	+\lim_{t\to 0} \frac{1}{t}
% 	\bigintssss_{-1}^{1}
% 	\Big(
% 	\exp{\Big((y+s\,v)^{\prime 2}\Big)}
% 	- \exp{\Big(y^{\prime 2}\Big)}
% 	\Big)\,dx                               \notag \\ 
% 	%-----------------------------------------------------   
% 	 & =  
% 	\underbrace{\lim_{t\to 0} \frac{1}{t}
% 		\bigintssss_{-1}^{1}
% 		\Big(
% 		s\,y\,v^\prime  
% 		+ s\,v\,y^\prime
% 		+ s^2\,v\,v^\prime
% 		\Big)
% 		\,dx}_{:=\mathcal{A}}                   \notag \\
% 	% 
% 	 & \qquad\qquad\qquad\qquad
% 	+
% 	\underbrace{\lim_{t\to 0} \frac{1}{t}
% 		\bigintssss_{-1}^{1}
% 		\Big(
% 		\exp{\Big((y+s\,v)^{\prime 2}\Big)}
% 		- \exp{\Big(y^{\prime 2}\Big)}
% 		\Big)\,dx}_{:=\mathcal{B}}
% 	% %-----------------------------------------------------   
% 	%  & = \mathcal{A} + \mathcal{B}.
% \end{align}

% Hence, it leads to
% \begin{align}
% 	\label{eq:sumAB}
% 	\therefore\quad
% 	\boxed{
% 		\delta F(y;v) = \mathcal{A} + \mathcal{B}.
% 	}
% \end{align}

% Next, we consider part $\mathcal{A}$, as follows
% \begin{align}
% 	\mathcal{A}
% 	 & = 
% 	\lim_{t\to 0} \frac{1}{t}
% 	\bigintssss_{-1}^{1}
% 	\Big(
% 	s\,y\,v^\prime  
% 	+ s\,v\,y^\prime
% 	+ s^2\,v\,v^\prime
% 	\Big)
% 	\,dx          
% 	%------------------------
% 	= 
% 	\lim_{t\to 0}
% 	\bigintssss_{-1}^{1}
% 	\Big(
% 	y\,v^\prime  
% 	+ v\,y^\prime
% 	+ s\,v\,v^\prime
% 	\Big)
% 	\,dx                      \notag \\
% 	%------------------------
% 	 & = 
% 	\lim_{t\to 0}
% 	\bigintssss_{-1}^{1}
% 	\Big(
% 	y\,v^\prime  
% 	+ v\,y^\prime
% 	\Big)
% 	\,dx
% 	+
% 	\lim_{t\to 0}
% 	\bigintssss_{-1}^{1}
% 	(s\,v\,v^\prime)
% 	\,dx                      \notag \\
% 	%------------------------
% 	 & = 
% 	\lim_{t\to 0}
% 	\bigintssss_{-1}^{1}
% 	\Big(
% 	y\,v^\prime  
% 	+ v\,y^\prime
% 	\Big)
% 	\,dx
% 	+
% 	\underbrace{\lim_{t\to 0}
% 		\left(
% 		s
% 		\bigintssss_{-1}^{1}
% 		(v\,v^\prime)
% 		\,dx
% 		\right)}_{=0}             \notag \\
% 	%------------------------
% 	 & = 
% 	\lim_{t\to 0}
% 	\bigintssss_{-1}^{1}
% 	\Big(
% 	y(x)\,v^\prime(x)
% 	+ v(x)\,y^\prime(x)
% 	\Big)
% 	\,dx          
% 	\label{eq:xgoesback}             \\
% 	%------------------------
% 	 & = 
% 	\lim_{t\to 0}
% 	\bigintssss_{-1}^{1}
% 	\Big(y(x)\,v(x)\Big)^\prime
% 	\,dx
% 	%------------------------
% 	= 
% 	\bigintssss_{-1}^{1}
% 	\frac{d}{dx}
% 	\Big(y(x)\,v(x)\Big)
% 	\,dx                \notag       \\
% 	%------------------------
% 	 & = 
% 	\bigintssss_{-1}^{1}
% 	\frac{d}{dx}
% 	\Big(y(x)\,v(x)\Big)
% 	\,dx
% 	%------------------------
% 	= 
% 	\Big(y(x)\,v(x)\Big)\Bigg|_{x=-1}^{x=1}
% 	= y(1)\,v(1) - y(-1)\,v(-1),
% \end{align}
% where $(x)$ has returned back 
% in \eqref{eq:xgoesback} again
% in order to get ready to compute the
% follow-up integral.
% Therefore, solution to $\mathcal{A}$ yields
% \begin{align}
% 	\label{eq:solutiontoA}
% 	\therefore\quad
% 	\boxed{
% 		\mathcal{A} = y(1)\,v(1) - y(-1)\,v(-1).
% 	}
% \end{align}

% Besides, the computation of part $\mathcal{B}$ goes as follows
% \begin{align}
% 	\mathcal{B}
% 	 & =
% 	\lim_{t\to 0} \frac{1}{t}
% 	\bigintssss_{-1}^{1}
% 	\Big(
% 	\exp{\Big((y+s\,v)^{\prime 2}\Big)}
% 	- \exp{\big(y^{\prime 2}\big)}
% 	\Big)\,dx  \notag \\
% 	%--------------------------------------
% 	 & =
% 	\lim_{t\to 0} \frac{1}{t}
% 	\bigintssss_{-1}^{1}
% 	\Big(
% 	\exp{
% 		\big(
% 		y^{\prime 2}
% 		+ 2\,s\,y^{\prime}\,v^{\prime}
% 		+ s^2\,v^{\prime 2}
% 		\big)}
% 	- \exp{\big(y^{\prime 2}\big)}
% 	\Big)\,dx  \notag \\
% 	%--------------------------------------
% 	 & =
% 	\lim_{t\to 0} \frac{1}{t}
% 	\bigintssss_{-1}^{1}
% 	\Big(
% 	\exp{\big(y^{\prime 2} \big)}
% 	\exp{\big(
% 		2\,s\,y^{\prime}\,v^{\prime}
% 		+ s^2\,v^{\prime 2}
% 		\big)}
% 	- \exp{\big(y^{\prime 2}\big)}
% 	\Big)\,dx  \notag \\
% 	%--------------------------------------
% 	 & =
% 	\lim_{t\to 0} \frac{1}{t}
% 	\bigintssss_{-1}^{1}
% 	\Bigg(
% 	\exp{\big(y^{\prime 2} \big)}
% 	\Big(
% 		\exp{\big(
% 			2\,s\,y^{\prime}\,v^{\prime}
% 			+ s^2\,v^{\prime 2}
% 			\big)}
% 		- 1
% 		\Big)
% 	\Bigg)\,dx \notag \\
% 	%--------------------------------------
% 	 & =
% 	\bigintssss_{-1}^{1}
% 	\Bigg(
% 	\exp{\big(y^{\prime 2} \big)}
% 	\underbrace{\lim_{t\to 0}
% 		\frac{
% 			\exp{\big(
% 				2\,s\,y^{\prime}\,v^{\prime}
% 				+ s^2\,v^{\prime 2}
% 				\big)}
% 			- 1}{t}}_{:=\mathcal{C}}
% 	\Bigg)\,dx ,
% 	\label{eq:Cexpression}
% \end{align}
% where a further result of $\mathcal{C}$ in 
% \eqref{eq:Cexpression} is computed by using l'Hôspital's rule, 
% as follows 
% \begin{align}
% 	\mathcal{C}
% 	 & =
% 	\lim_{t\to 0}
% 	\frac{
% 		\exp{\big(
% 			2\,s\,y^{\prime}\,v^{\prime}
% 			+ s^2\,v^{\prime 2}
% 			\big)}
% 		- 1}{t}                            \notag \\
% 	%------------------------
% 	 & =
% 	\lim_{t\to 0}
% 	\Big(
% 	\left(
% 	2\,y^{\prime}\,v^{\prime}
% 	+ 2\,s\,v^{\prime 2}
% 	\right)
% 	\exp{\big(
% 		2\,s\,y^{\prime}\,v^{\prime}
% 		+ s^2\,v^{\prime 2}
% 		\big)}
% 	\Big)                              \notag \\
% 	%------------------------
% 	 & = 2\,y^{\prime}\,v^{\prime} \notag     \\
% 	 & = 2\,y^{\prime}(x)\,v^{\prime}(x).
% 	\label{eq:Cresult}
% \end{align}
% Substitution of 
% \eqref{eq:Cresult} into 
% \eqref{eq:Cexpression}
% leads to 
% \begin{align}
% 	\label{eq:finalsolB}
% 	\therefore\quad
% 	\boxed{
% 	\mathcal{B} = 
% 	2\bigintssss_{-1}^{1}
% 	\exp{\left(y^{\prime 2}(x)\right)}
% 	\,y^{\prime}(x)\,v^{\prime}(x)\,dx.
% 	}
% \end{align}
% Finally, by inserting 
% \eqref{eq:solutiontoA}
% and
% \eqref{eq:finalsolB}
% into
% \eqref{eq:sumAB}
% the final result yields
% \begin{align}
% 	\therefore\quad\boxed{
% 	\delta F(y;v)
% 	=
% 	y(1)\,v(1) - y(-1)\,v(-1)
% 	+
% 	2\bigintssss_{-1}^{1}
% 	\,y^{\prime}(x)\,v^{\prime}(x)
% 	\exp{\left(y^{\prime 2}(x)\right)}
% 	\,dx.
% 	}
% \end{align}

% \newpage
% %------------------------------------------------------------------------------
% \section{Step-by-step with A4-SRU00}
% \begin{exampleboxed}
% 	Examine the following initial value problem (IVP)
% 	\begin{align}
% 		\label{eq:givenODE3rdorder}
% 		y^{\prime\prime\prime}(t)+y^{\prime}(t)=t\,y(t),
% 	\end{align}
% 	with initial conditions (IC) given as follows
% 	\begin{align}
% 		\label{eq:ICfromODE}
% 		y(0)=0,\quad y^{\prime}(0)=1,\quad y^{\prime\prime}(0)=2.
% 	\end{align}
% 	\begin{enumerate}[i)]
% 		\item Transform the above third order
% 		      ordinary differential equation (ODE) 
% 		      to a system of first order ODEs.
% 		\item Solve the system i) numerically by using explicit Euler method with 2 steps.
% 	\end{enumerate}
% \end{exampleboxed}

% Approach:
% \begin{enumerate}[i)]
% 	\item We need to transform the third order ODE
% 	      given in $\eqref{eq:givenODE3rdorder}$
% 	      into a system of first order ODEs.
% 	      \begin{questionboxed} 
% 		      But.. how many ODEs shall we need in total 
% 		      for this system of first order ODEs?\\ 
% 		      \textbf{Ans}: 3.
% 	      \end{questionboxed}
% 	      \begin{observationboxed}
% 		      Since the highest order of derivative in 
% 		      $\eqref{eq:givenODE3rdorder}$ is 3, 
% 		      i.e. as we can see there are three $'$ from $y'''(t)$,
% 		      the given ODE is called third order ODE.
% 		      Besides, the lowest order in 
% 		      $\eqref{eq:givenODE3rdorder}$ is 0, 
% 		      i.e. as we can see from $y(t)$.
% 		      Therefore, we shall need
% 		      a system of \textbf{three} first order ODEs
% 		      for the given third order ODE in 
% 		      $\eqref{eq:givenODE3rdorder}$.
% 	      \end{observationboxed}
% 	      \begin{questionboxed} 
% 		      Which variable from $\eqref{eq:givenODE3rdorder}$
% 		      shall we need to take and define,
% 		      so that we are able to transform 
% 		      $\eqref{eq:givenODE3rdorder}$
% 		      into a system of first order ODEs
% 		      corresponding to 
% 		      the given initial conditions given in 
% 		      \eqref{eq:ICfromODE}, 
% 		      e.g. is it ideal to define $y'''(t) := u_{3}(t)$?\\
% 		      \textbf{Ans}: $y(t)$, $y'(t)$ and $y''(t)$; 
% 		      and it is not ideal to go with 
% 		      $y'''(t) := u_{3}(t)$.
% 	      \end{questionboxed}
% 	      \begin{observationboxed}
% 		      The three initial conditions given
% 		      $y(0)$, $y'(0)$ and $y''(0)$, 
% 		      as shown in \eqref{eq:ICfromODE},
% 		      give rise to the fact that 
% 		      we shall need to define three new variables
% 		      for $y(t)$, $y'(t)$ and $y''(t)$.
% 		      This choice is ideal for the ODE with 
% 		      the given ICs
% 		      because we will be able to apply
% 		      these ICs directly for the new system of
% 		      three ODEs.
% 	      \end{observationboxed}

% 	      Therefore, we shall need three first order ODEs 
% 	      for the new system. Besides, 
% 	      the transformation is done by setting 
% 	      new variables for $y(t)$, $y'(t)$ and $y''(t)$.

% 	      Transformation: We now need to set
% 	      $y(t)$, $y'(t)$ and $y''(t)$
% 	      to new variables, 
% 	      let's say $u_0(t)$, $u_1(t)$, and $u_2(t)$,
% 	      respectively, 
% 	      as follows
% 	      \begin{align}
% 		      y(t)                & := u_0(t),  
% 		      \label{eq:newvariablesu0}          \\
% 		      y^{\prime}(t)       & := u_1(t),  
% 		      \label{eq:newvariablesu1}          \\
% 		      y^{\prime\prime}(t) & := u_2(t). 
% 		      \label{eq:newvariablesu2}
% 	      \end{align}

% 	      Then, by taking the derivative of 
% 	      the new variable $u_0(t)$, $u_1(t)$ and $u_2(t)$,
% 	      as newly defined in
% 	      \eqref{eq:newvariablesu0},
% 	      \eqref{eq:newvariablesu1} and
% 	      \eqref{eq:newvariablesu2},
% 	      we obtain the following relations
% 	      \begin{align}
% 		      \boxed{
% 			      \begin{aligned}
% 				      u_0^{\prime}(t) 
% 				       & \stackrel{\eqref{eq:newvariablesu0}}{=}
% 				      y'(t)  
% 				      \stackrel{\eqref{eq:newvariablesu1}}{=}
% 				      u_1(t),                                    \\
% 				      u_1^{\prime}(t) 
% 				       & \stackrel{\eqref{eq:newvariablesu1}}{=}
% 				      y''(t)  
% 				      \stackrel{\eqref{eq:newvariablesu2}}{=}
% 				      u_2(t),                                    \\
% 				      u_2^{\prime}(t) 
% 				       & 
% 				      \stackrel{\eqref{eq:newvariablesu2}}{=}
% 				      y'''(t)
% 				      \stackrel{\eqref{eq:givenODE3rdorder}}{=}
% 				      t\,y(t)-y'(t)
% 				      \stackrel{
% 					      \stackrel
% 					      {\mbox{\scriptsize $\eqref{eq:givenODE3rdorder}$}}
% 					      {\eqref{eq:givenODE3rdorder}}
% 				      }{=}
% 				      t\,u_0(t)-u_1(t),
% 			      \end{aligned}
% 		      }
% 	      \end{align}
% 	      which is summarized, 
% 	      together with keeping in mind about 
% 	      the order of the index $i$ from 
% 	      $u_{i}(t)|_{i=0,1,2}$, 
% 	      as follows 
% 	      \begin{align}
% 		      \begin{array}{ccccc}
% 			      u'_{0}(t) & = &             & u_{1}(t)  &          \\
% 			      u'_{1}(t) & = &             &           & u_{2}(t) \\
% 			      u'_{2}(t) & = & t\,u_{0}(t) & -u_{1}(t) & 
% 		      \end{array}
% 	      \end{align}
% 	      which gives rise to the following system 
% 	      \begin{align}
% 		      \begin{array}{ccccc}
% 			      u'_{0}(t) & = & 0\,u_{0}(t) & 1\,u_{1}(t)  & 0\,u_{2}(t) \\
% 			      u'_{1}(t) & = & 0\,u_{0}(t) & 0\,u_{1}(t)  & 1\,u_{2}(t) \\
% 			      u'_{2}(t) & = & t\,u_{0}(t) & -1\,u_{1}(t) & 0\,u_{2}(t)
% 		      \end{array}
% 	      \end{align}
% 	      from where we can write the system in form of 
% 	      \begin{align}
% 		      \Bu' = A\Bu	
% 	      \end{align}
% 	      as follows
% 	      \begin{align}
% 		      \label{eq:newsystem}
% 		      \boxed{
% 			      \begin{aligned}
% 				      \underbrace{
% 				      \begin{pmatrix} u_{0}(t)\\u_{1}(t)\\u_{2}(t) \end{pmatrix}^{\prime}
% 				      }_{:=\Bu^{\prime}}
% 				      =
% 				      \underbrace{
% 					      \begin{pmatrix}
% 						      0 & 1 & 0 \\ 0 & 0 & 1 \\ t & -1 & 0
% 					      \end{pmatrix}
% 				      }_{:=A}
% 				      \underbrace{
% 					      \begin{pmatrix} u_{0}(t)\\u_{1}(t)\\u_{2}(t) \end{pmatrix}
% 				      }_{:=\Bu}
% 			      \end{aligned}
% 		      }
% 	      \end{align}
% 	      where the relation between $A$ and $\Bu$ in $A\Bu$
% 	      is matrix-vector multiplication. Besides,
% 	      by substituting 
% 	      \eqref{eq:newvariablesu0}
% 	      \eqref{eq:newvariablesu1}
% 	      and 
% 	      \eqref{eq:newvariablesu2}
% 	      into 
% 	      \eqref{eq:ICfromODE}, respectively,
% 	      we obtain the initial conditions 
% 	      for the new system in 
% 	      \eqref{eq:newsystem}
% 	      \begin{align}
% 		      \boxed{
% 			      \begin{aligned}
% 				      u_0(0)=0, \\ 
% 				      u_1(0)=1, \\ 
% 				      u_2(0)=2,
% 			      \end{aligned}
% 		      }
% 	      \end{align}
% 	      which form an equivalent system of three first-order ODEs.
% 	      \newpage
% 	\item We need to solve the
% 	      system of first order ODEs obtained in i)
% 	      numerically by using explicit Euler method.
% 	      \begin{summaryboxed}
% 		      Given the system of first order ODEs as follows
% 		      \begin{align}
% 			      \label{eq:newBuprime}
% 			      \Bu'(t) = A\Bu(t)
% 		      \end{align}
% 		      where the vector $\Bu$ is defined as 
% 		      $\Bu := (u_{0},\, u_{1},\, u_{2})^{\top}$,
% 		      and the matrix $A$ is
% 		      \begin{align}
% 			      A =
% 			      \begin{pmatrix}
% 				      0 & 1 & 0 \\ 0 & 0 & 1 \\ t & -1 & 0
% 			      \end{pmatrix}
% 		      \end{align}
% 		      with initial conditions 
% 		      \begin{align}
% 			      u_0(0)=0,\quad u_1(0)=1,\quad u_2(0)=2.
% 		      \end{align}
% 		      Solve this system numerically 
% 		      by using the explicit Euler method.
% 	      \end{summaryboxed}

% 	      Approach:
% 	      We need to approximate $\Bu'(t)$ arising in 
% 	      \eqref{eq:newBuprime} by some numerical schemes, 
% 	      e.g. explicit Euler method.
% 	      %   \begin{questionboxed} 
% 	      %       But.. how does the explicit Euler formula look like? 
% 	      %       And how can it be derived exactly?\\
% 	      %       Ans: Exploit the derivation in 
% 	      %       \Cref{eq:observationboxedtaylor}.
% 	      %   \end{questionboxed}
% 	      \begin{observationboxed}
% 		      \label{eq:observationboxedtaylor}
% 		      We consider Taylor expansion 
% 		      for $\Bu(t)$ around point $t=t_{0}$
% 		      \begin{align}
% 			      \label{eq:Taylorforut}
% 			      \Bu(t)
% 			      = \Bu(t_{0}) + \Bu'(t_{0})(t-t_{0}) 
% 			      + \mathcal{O}((t-t_{0})^2),
% 		      \end{align}
% 		      which, for the sake of convergence, 
% 		      requires $h:=t-t_{0}$ to be small.
% 		      Since $h:=t-t_{0}$, we equally have 
% 		      $t=h+t_{0}$. Substitution
% 		      of $h:=t-t_{0}$ and $t=t_{0}+h$
% 		      to \eqref{eq:Taylorforut}
% 		      leads to the following expression
% 		      \begin{align}
% 			      \label{eq:Taylorforutnew}
% 			      \Bu(t_{0}+h)
% 			      = \Bu(t_{0}) + h\Bu'(t_{0})
% 			      + \mathcal{O}(h^2),
% 		      \end{align}
% 		      which is the new expression of Taylor expansion 
% 		      in terms of point $t_{0}$,
% 		      around which the function $\Bu(t)$
% 		      is approximated by the Taylor expansion,
% 		      instead of the classical expression 
% 		      in terms of the variable $t$
% 		      as seen in \eqref{eq:Taylorforut}.
% 		      Since point $t_{0}$, 
% 		      although it implied a fixed point,
% 		      arising in \eqref{eq:Taylorforutnew}
% 		      can be chosen arbitrarily, 
% 		      we can, therefore, substitute it for $t$, 
% 		      as follows
% 		      \begin{align}
% 			      \label{eq:Taylorforutnewnozero}
% 			      \Bu(t+h)
% 			      = \Bu(t) + h\Bu'(t)
% 			      + \mathcal{O}(h^2),
% 		      \end{align}
% 		      Now, the expression \eqref{eq:Taylorforutnew}
% 		      gives rise to a useful expression
% 		      to approximate $\Bu'(t)$ numerically, as follows
% 		      \begin{align}
% 			      \label{eq:Taylorforutnewnozerofinal}
% 			      \therefore\quad
% 			      \boxed{
% 				      \Bu'(t) 
% 				      \approx \frac{\Bu(t+h) - \Bu(t)}{h}.
% 			      }
% 		      \end{align}
% 	      \end{observationboxed}
% 	      %   The ODE in \eqref{eq:newBuprime} 
% 	      %   has a general form 
% 	      %   \begin{align}
% 	      % 	\Bu'(t) = f(t,\Bu(t))
% 	      %   \end{align}
% 	      %   Definition:
% 	      %   \begin{align}
% 	      %       \Bu(t+h) & : = \Bu_{j+1} \\
% 	      %       \Bu(t)   & : = \Bu_{j}
% 	      %   \end{align}
% 	      Substitution of 
% 	      \eqref{eq:Taylorforutnewnozerofinal}
% 	      into \eqref{eq:newBuprime} yields
% 	      \begin{align}
% 		      \frac{\Bu(t+h) - \Bu(t)}{h}
% 		      = A\Bu(t)
% 	      \end{align}
% 	      which leads to the formula of explicit Euler method 
% 	      \begin{align}
% 		      \label{eq:Eulerformulaformu}
% 		      \boxed{
% 			      \Bu^{(j+1)} = \Bu^{(j)} + h\,A\Bu^{(j)},
% 		      }
% 	      \end{align}
% 	      where $\Bu^{(j)}$ is the numerical solution by
% 	      Euler method after $j$ steps,
% 	      and components of $\Bu^{(j)}$ is known as 
% 	      \begin{align}
% 		      \Bu^{(j)}
% 		      = 
% 		      \begin{pmatrix} 
% 			      u_0 \\u_1 \\u_2
% 		      \end{pmatrix}^{(j)},
% 	      \end{align}
% 	      which, together with 
% 	      \eqref{eq:Eulerformulaformu},
% 	      leads to the following expressions
% 	      \begin{align}
% 		      \therefore\quad
% 		      \boxed{
% 			      \begin{aligned}
% 				      \Bu^{(0)} & =
% 				      \begin{pmatrix} 
% 					      u_0^{(0)} \\u_1^{(0)} \\u_2^{(0)}
% 				      \end{pmatrix}
% 				      =
% 				      \begin{pmatrix}
% 					      0 \\1\\2
% 				      \end{pmatrix},              \\
% 				      \phantom{a}                              \\
% 				      %------------------------------
% 				      \Bu^{(1)} & =
% 				      \begin{pmatrix}
% 					      u_0^{(0)} \\u_1^{(0)} \\u_2^{(0)}
% 				      \end{pmatrix}
% 				      +h
% 				      \underbrace{
% 					      \begin{pmatrix}
% 						      0 & 1 & 0 \\ 0 & 0 & 1 \\ t & -1 & 0
% 					      \end{pmatrix}
% 				      }_{=A}
% 				      \begin{pmatrix}
% 					      u_0^{(0)} \\u_1^{(0)} \\u_2^{(0)}
% 				      \end{pmatrix}
% 				      \\
% 				                & =
% 				      \begin{pmatrix}
% 					      0 \\1\\2
% 				      \end{pmatrix}
% 				      +h\begin{pmatrix}
% 					      1 \\2\\0-1
% 				      \end{pmatrix}
% 				      \\
% 				                & =\begin{pmatrix}
% 					      h \\1+2h\\2-h
% 				      \end{pmatrix}, \\
% 				      \phantom{a}                              \\
% 				      %------------------------------
% 				      \Bu^{(2)} & =
% 				      \begin{pmatrix}
% 					      u_0^{(1)} \\u_1^{(1)} \\u_2^{(1)}
% 				      \end{pmatrix}
% 				      +h
% 				      \underbrace{
% 					      \begin{pmatrix}
% 						      0 & 1 & 0 \\ 0 & 0 & 1 \\ t & -1 & 0
% 					      \end{pmatrix}
% 				      }_{=A}
% 				      \begin{pmatrix}
% 					      u_0^{(1)} \\u_1^{(1)} \\u_2^{(1)}
% 				      \end{pmatrix}
% 				      \\
% 				                & =
% 				      \begin{pmatrix}
% 					      h \\1+2h\\2-h
% 				      \end{pmatrix}
% 				      +h\begin{pmatrix}
% 					      1+2h \\2-h\\h\cdot h-(1+2h)
% 				      \end{pmatrix}             \\
% 				                & 
% 				      =\begin{pmatrix}
% 					      2h(1+h) \\1+4h-h^2\\2-2h-2h^2+h^3
% 				      \end{pmatrix}.
% 			      \end{aligned}
% 		      }
% 	      \end{align}
% \end{enumerate}
% \newpage
% \begin{align}
% 	\delta F(y;v)
% 	 & :=  \lim_{t\to 0}\frac{F(y+sv)-F(y)}{t} \\
% 	 & =  \lim_{t\to 0} \frac{1}{t}
% 	\int_{-1}^{1}
% 	\Big(
% 	\left(y+sv\right)(x) \left(y+sv\right)^\prime(x)
% 	+ \exp{(y+sv)^{\prime 2}(x)}
% 	\Big)\,dx                                  \\
% 	% 
% 	 & \qquad\qquad\qquad\qquad\qquad\qquad
% 	-\lim_{t\to 0} \frac{1}{t}
% 	\int_{-1}^{1}
% 	\Big(
% 	y(x)y'(x) + \exp{y^{\prime 2}(x)}
% 	\Big)\,dx                                  \\
% 	% 
% 	 & =  \lim_{t\to 0} \frac{1}{t}
% 	\int_{-1}^{1}
% 	\Big(
% 	y(x)y'(x) + sy(x)v'(x) + sv(x)y'(x) + s^2v(x)v'(x) - y(x)y'(x)
% 	\Big)\,dx                                  \\
% 	% 
% 	 & \qquad\qquad\qquad\qquad\qquad\qquad
% 	-\lim_{t\to 0} \frac{1}{t}
% 	\int_{-1}^{1}
% 	\Big(
% 	\exp{y^{\prime 2}(x)}
% 	- \exp{(y+sv)^{\prime 2}(x)}
% 	\Big)\,dx                                  \\
% 	% 
% 	%  & =  \lim_{t\to 0}
% 	% \int_{-1}^{1}
% 	% \Big(
% 	% y(x)\,v'(x) + v(x)\,y'(x) + s\,v(x)\,v'(x)
% 	% \Big)\,dx                                                    \\
% 	%
% 	%  & \qquad\qquad\qquad\qquad\qquad\qquad
% 	% -\lim_{t\to 0} \frac{1}{t}
% 	% \int_{-1}^{1}
% 	% \Big(
% 	% \exp{y^{\prime 2}(x)}
% 	% - \exp{(y+sv)^{\prime 2}(x)}
% 	% \Big)\,dx                                                    \\
% 	%
% 	 & = \lim_{s\rightarrow 0}
% 	\int_{-1}^1
% 	\Big(
% 	y(x)\,v^\prime(x)+y^\prime(x)\,v(x) + s\,v(x)\,v^\prime(x)
% 	\Big)
% 	\,dx                                       \\
% 	% 
% 	 & \qquad\qquad\qquad\qquad\qquad\qquad
% 	+\lim_{t\to 0} \frac{1}{t}
% 	\int_{-1}^{1}
% 	\Big(
% 	\exp{\big(y(x)+s\,v(x)\big)^{\prime 2}}-\exp{y(x)^{\prime 2}}
% 	\Big)
% 	\,dx                                       \\
% 	% 
% 	 & = \int_{-1}^1
% 	\lim_{s\rightarrow 0}
% 	\Big(
% 	y(x)\,v^\prime(x)
% 	+ y^\prime(x)\,v(x) + s\,v(x)\,v^\prime(x)
% 	\Big)\,dx                                  \\
% 	 & \qquad\qquad\qquad
% 	+ \int_{-1}^1
% 	\exp{y^{\prime 2}(x)}
% 	\lim_{t\to 0}
% 	\underbrace{
% 		\left(
% 		\frac{
% 			\exp{\Big( 2 y^\prime(x)\,v^\prime(x)\,s
% 				+ v^{\prime 2}(x)\,s^2\Big)}-1
% 		}{t}
% 		\right)
% 	}
% 	_{\text{l'Hospital}}
% 	\,dx                                       \\
% 	 & = \int_{-1}^1
% 	\Big(
% 	y(x)\,v^\prime(x)
% 	+ y^\prime(x)\,v(x)
% 	\Big)\,dx                                  \\
% 	 & \
% 	+ \int_{-1}^1
% 	\exp{y^{\prime 2}(x)}
% 	\lim_{t\to 0}
% 	\left(
% 	\left(
% 	2 y^\prime(x)\,v^\prime(x) + v^{\prime 2}(x)\,s
% 	\right)
% 	\exp{\Big( 2 y^\prime(x)\,v^\prime(x)\,s
% 		+ v^{\prime 2}(x)\,s^2\Big)}
% 	\right)
% 	\,dx                                       \\
% 	 & = \int_{-1}^1
% 	\Big(
% 	(y\,v)^\prime(x)
% 	+ 2 y^\prime(x)\,v^\prime(x)\,\exp{y^{\prime 2}(x)}
% 	\Big)\,dx.
% 	% = y(1)\,v(1) - y(-1)\,v(-1)
% 	% + 2 \int_{-1}^1
% 	% y^\prime(x)\,v^\prime(x)\,\exp{y^{\prime 2}(x)}
% 	% \,dx.
% \end{align}
% Daher
% \begin{align}
% 	\therefore\quad\boxed{
% 		\delta F(y;v)
% 		= y(1)\,v(1) - y(-1)\,v(-1)
% 		+ 2 \int_{-1}^1
% 		y^\prime(x)\,v^\prime(x)\,\exp{y^{\prime 2}(x)} \,dx
% 	}
% \end{align}

% %------------------------------------------------------------------------------
% \section{Numerics: FDM $\cdot$ BC $\cdot$ Derivation of $A_{h}$ $\&$ $b_{h}$}
% \begin{example}
% 	Discretization of the following convection-diffusion problem:
% 	\begin{align}
% 		-u''(x) -5u'(x) = 3,
% 		\quad \text{for}\ x\in (0,1)
% 	\end{align}
% 	with boundary conditions $u(0)=u(1)=1.$
% \end{example}
% \inputfig{floats/discretize}{discretize}
% Approach:
% Central method for $u''(x)$ is obtained as follows
% \begin{align}
% 	u''(x)=\frac{u(x+h)-2u(x)+u(x-h)}{h^2}+{\cal O}(h^2).
% \end{align}
% Forward method for $u'(x)$ is obtained as follows
% \begin{align}
% 	u'(x)=\frac{u(x+h) -u(x)}{h} + {\cal O}(h).
% \end{align}
% which leads to the system $A_h u_h=b_h$ where $A_h$ and $b_h$ are recognized as follows
% \begin{align}
% 	A_h & =\frac{1}{h^2}\left(
% 	\begin{array}{cccccc}
% 			2+5h & -1-5h  &        &        &        &       \\
% 			-1   & 2+5h   & -1-5h  &        &        &       \\ 
% 			     & \ddots & \ddots & \ddots &        &       \\ 
% 			     &        & \ddots & \ddots & \ddots &       \\ 
% 			     &        &        & -1     & 2+5h   & -1-5h \\ 
% 			     &        &        &        & -1     & 2+5h
% 		\end{array}\right)
% 	\in\mathbb{R}^{(n-2)\times (n-2)}, \\
% 	b_h & = 
% 	\begin{pmatrix}
% 		\displaystyle
% 		3+\frac{1}{h^2} \\ 3\\  \vdots \\ 3 \\
% 		\displaystyle
% 		3+\frac{1}{h^2}+\frac{5}{h}
% 	\end{pmatrix}
% 	\in\mathbb{R}^{(n-2)}.
% \end{align}

% %------------------------------------------------------------------------------
% \newpage
% \inputfig{floats/discretize}{discretize}

% %------------------------------------------------------------------------------
% \newpage
% \inputfig{floats/discretize}{discretize}

% %------------------------------------------------------------------------------
% \newpage
% \section{Analysis: Spectral theory $\cdot$ Laplace operator application}
% \newpage
% \begin{example}
% 	Examine the following problem
% 	\begin{align*}
% 		\partial_t u(x,t) & =\partial_{xx}u(x,t)+a, &  & x\ \in\ (0,1), t>0, \\
% 		u(0,t)            & =0,                     &  & t>0,                \\
% 		u(1,t)            & =0,                     &  & t>0,                \\
% 		u(x,0)            & = \sin(\pi x),          &  & x\, \in\, (0,1) ,
% 	\end{align*}
% 	with constant source term $a\in\mathbb{R}$.
% 	\begin{enumerate}
% 		\item Compute the Eigenvalues and the normalized Eigenfunctions from $\partial_{xx}$.
% 		\item Develop the source term $a\in\mathbb{R}$ in these Eigenfunctions.
% 		\item Compute the solution of the problem with the Eigenfunction ansatz.
% 	\end{enumerate}
% \end{example}

% \newpage
% (Example 2 cont.)

% \newpage
% (Example 2 cont.)

% \begin{enumerate}
% 	\item Die zum Problem geh"orenden Eigenfunktionen und Eigenwerte erf"ullen
% 	      \begin{align*}
% 		      \varphi^{\prime\prime}(x)+\lambda\varphi(x) & =0, \quad x\in (0,1) \\
% 		      \varphi(0)=\varphi(1)                       & =0
% 	      \end{align*}
% 	      \begin{align*}
% 		      \Rightarrow \quad \varphi(x) = C_1\sin(\sqrt{\lambda}x) + C_2 \cos(\sqrt{\lambda}x)
% 	      \end{align*}
% 	      Mit den Randbedingungen f"ur $\varphi$ erh"alt man 
% 	      \begin{align*}
% 		      0 = \varphi(0) & = C_2 \, \Rightarrow \, C_2 = 0                                                       \\
% 		      0 = \varphi(1) & = C_1\sin(\sqrt{\lambda}x) \, \Rightarrow \, \lambda_k = (k\pi)^2, \, k\in \mathbb{N}
% 	      \end{align*}
% 	      \begin{align*}
% 		      \Rightarrow \varphi_k(x) = C_1\sin(k\pi x) , \, k \in \mathbb{N}
% 	      \end{align*}
% 	      $C_1$ erh"alt man durch Normierung
% 	      \begin{align*}
% 		      1 & \stackrel{!}{=} \int_0^1 \varphi_k(x)^2\, dx = C_1^2 \int_0^1 \sin^2(k\pi x) \, dx                \\
% 		        & = C_1^2 \left[ \left. \frac x 2 - \frac{\sin(2k\pi x)}{4k\pi}\right|^1_0 \right] = C_1^2\frac 1 2 \\
% 		      \Rightarrow C_1 = \sqrt{2}
% 	      \end{align*}
% 	      damit lauten die Eigenwerte und normierten Eigenfunktionen 
% 	      \begin{align*}
% 		      \varphi_k(x)               & =\sqrt{2}\sin(k\pi x)                  \\
% 		      \text{und} \quad \lambda_k & =(k\pi)^2 \text{ f"ur } k\in\mathbb{N}
% 	      \end{align*}

% 	\item Die Entwicklung des konstanten Quellterms $a(x,t)=a$  in Eigenfunktionen lautet
% 	      \begin{align*}
% 		      a & = \sum^{\infty}_{k=1}\beta_k\varphi_k(x)         \\
% 		        & = \sum^{\infty}_{k=1}\beta_k\sqrt{2}\sin(k\pi x)
% 	      \end{align*}
% 	      Die Koeffizienten berechnet man durch
% 	      \begin{align*}
% 		      \beta_k & =\int^1_0 a\sqrt{2}\sin(k\pi x)dx                                         \\
% 		              & = a \sqrt{2} \left[ \left. -\frac{\cos(k \pi x)}{k\pi}\right|^1_0 \right] \\
% 		              & =\frac{a\sqrt{2}}{k\pi}(1-\cos(k\pi))=\begin{cases}\frac{2a\sqrt{2}}{k\pi} & , k\text{ ungerade} \\
% 			      0                       & , \text{sonst}
% 		      \end{cases}
% 	      \end{align*}

% 	\item F"ur die L"osung $u$ wird der Eigenfunktionen-Ansatz
% 	      \begin{align*}
% 		      u(x,t)=\sum^{\infty}_{k=1}\alpha_k(t)\varphi_k(x)
% 	      \end{align*}
% 	      gew"ahlt, mit zeitabh"angigen Koeffizienten $\alpha_k(t), \, t>0, k\in \mathbb{N}$. \\
% 	      Einsetzen der Entwicklungen von $u(x,t)$ und $a$ in die Differentialgleichung liefert
% 	      \begin{align*}
% 		                           & (\sum^{\infty}_{k=1}\alpha_k(t)\varphi_k(x))_t=(\sum^{\infty}_{k=1}\alpha_k(t)\varphi_k(x))_{xx}+\sum^{\infty}_{k=1}\beta_k\varphi_k(x) \\
% 		      \Leftrightarrow\quad & \sum^{\infty}_{k=1}\alpha'_k(t)\varphi_k(x)=-\sum^{\infty}_{k=1}\alpha_k(t)\lambda_k\varphi_k(x)+\sum^{\infty}_{k=1}\beta_k\varphi_k(x) \\
% 		      \Leftrightarrow\quad & \sum^{\infty}_{k=1}(\alpha'_k(t)+\lambda_k\alpha_k(t)-\beta_k)\varphi_k(x)=0
% 	      \end{align*}
% 	      Wir erhalten eine gew"ohnliche Differentialgleichung f"ur die Koeffizienten $\alpha_k(t)$
% 	      \begin{align*}
% 		      \alpha'_k(t) & =-\lambda_k\alpha_k(t)+\beta_k
% 	      \end{align*}
% 	      mit der allgemeinen L"osung 
% 	      \begin{align*}
% 		      \alpha_k(t) & =\frac{\beta_k}{\lambda_k}+C_k e^{-\lambda_k t}.
% 	      \end{align*}
% 	      Die Konstante $C_k$ wird durch die Anfangsbedingung f"ur $\alpha_k(0)$ bestimmt. Dazu betrachten wir die Anfangsbedingung f"ur $u$:
% 	      \begin{align*}
% 		      u(x,0)=\sum^{\infty}_{k=1}\alpha_k(0)\sqrt{2}\sin(k\pi x)\stackrel{!}{=} \sin(\pi x) \\
% 		      \Rightarrow \alpha_1(0)\stackrel{!}{=}\frac{1}{\sqrt{2}} \,\text{ und } \, \alpha_k(0) = 0 \ \forall\ k > 1
% 	      \end{align*}
% 	      Also gilt f"ur $k=1$
% 	      \begin{align*}
% 		      \alpha_1(0)     & = \frac{\beta_1}{\lambda_1}+C_1  \stackrel{!}{=}\frac{1}{\sqrt{2}}                                                            \\
% 		      \Rightarrow C_1 & =\frac{1}{\sqrt{2}} -\frac{\beta_1}{\lambda_1} = \frac{1}{\sqrt{2}}-\frac{2a\sqrt{2}}{\pi^3} = \frac{\pi^3-4a}{\sqrt{2}\pi^3}
% 	      \end{align*}
% 	      und f"ur $k>1$
% 	      \begin{align*}
% 		      \alpha_k(0)     & =\frac{\beta_k}{\lambda_k}+C_k\stackrel{!}{=}0           \\
% 		      \Rightarrow C_k & =-\frac{\beta_k}{\lambda_k} = \begin{cases}-\frac{2a\sqrt{2}}{k^3\pi^3} & , k\text{ ungerade} \\
% 			      0                            & , \text{sonst}
% 		      \end{cases}
% 	      \end{align*}
% 	      Die L"osung $u$ lautet daher insgesamt 
% 	      \begin{align*}
% 		      u(x,t) & =\alpha_1(t)\sqrt{2}\sin(\pi x) + \sum^{\infty}_{k=2}(\frac{\beta_k}{\lambda_k} + C_ke^{-\lambda_k t})\sqrt{2}\sin(k\pi x)                                                                                   \\
% 		             & = \left(\frac{ 4a}{\pi^3} + (1-\frac{4a}{\pi^3})e^{-\pi^2t}\right)\sin(\pi x) +\sum^{\infty}_{k=2 \atop k\text{ ungerade}} \left( \frac{4a}{k^3\pi^3} - \frac{4a}{k^3\pi^3}e^{-k^2\pi^2t}\right)\sin(k\pi x)
% 	      \end{align*}
% \end{enumerate}

% %------------------------------------------------------------------------------
% \newpage
% \begin{example}
% 	Examine the following problem
% 	\begin{align*}
% 		\partial_t u(x,t) & =\partial_{xx}u(x,t)+\sin(\pi x), &  & x\ \in\ (0,1), t>0, \\
% 		u(0,t)            & =0,                               &  & t>0,                \\
% 		u(1,t)            & =0,                               &  & t>0,                \\
% 		u(x,0)            & = a,                              &  & x\, \in\, (0,1) ,
% 	\end{align*}
% 	with constant IC $a\in\mathbb{R}$.
% 	\begin{enumerate}
% 		\item Compute the Eigenvalues and the normalized Eigenfunctions from $\partial_{xx}$.
% 		\item Develop the IC and source term in these Eigenfunctions.
% 		\item Compute the solution of the problem with the Eigenfunction ansatz.
% 	\end{enumerate}
% \end{example}

% \begin{enumerate}
% 	\item Die zum Problem geh"orenden Eigenfunktionen und Eigenwerte erf"ullen
% 	      \begin{align*}
% 		      \varphi^{\prime\prime}(x)+\lambda\varphi(x) & =0, \quad x\in (0,1) \\
% 		      \varphi(0)=\varphi(1)                       & =0
% 	      \end{align*}
% 	      \begin{align*}
% 		      \Rightarrow \quad \varphi(x) = C_1\sin(\sqrt{\lambda}x) + C_2 \cos(\sqrt{\lambda}x)
% 	      \end{align*}
% 	      Mit den Randbedingungen f"ur $\varphi$ erh"alt man 
% 	      \begin{align*}
% 		      0 = \varphi(0) & = C_2 \, \Rightarrow \, C_2 = 0                                                      \\
% 		      0 = \varphi(1) & = C_1\sin(\sqrt{\lambda}) \, \Rightarrow \, \lambda_k = (k\pi)^2, \, k\in \mathbb{N}
% 	      \end{align*}
% 	      \begin{align*}
% 		      \Rightarrow \varphi_k(x) = C_1\sin(k\pi x) , \, k \in \mathbb{N}
% 	      \end{align*}
% 	      $C_1$ erh"alt man durch Normierung
% 	      \begin{align*}
% 		      1 & \stackrel{!}{=} \int_0^1 \varphi_k(x)^2\, dx = C_1^2 \int_0^1 \sin^2(k\pi x) \, dx                \\
% 		        & = C_1^2 \left[ \left. \frac x 2 - \frac{\sin(2k\pi x)}{4k\pi}\right|^1_0 \right] = C_1^2\frac 1 2 \\
% 		      \Rightarrow C_1 = \sqrt{2}
% 	      \end{align*}
% 	      damit lauten die Eigenwerte und normierten Eigenfunktionen 
% 	      \begin{align*}
% 		      \varphi_k(x)               & =\sqrt{2}\sin(k\pi x)                  \\
% 		      \text{und} \quad \lambda_k & =(k\pi)^2 \text{ f"ur } k\in\mathbb{N}
% 	      \end{align*}

% 	\item Die Entwicklung der konstanten Anfangsbedingung $a(x,t)=a$  in Eigenfunktionen lautet
% 	      \begin{align*}
% 		      a & = \sum^{\infty}_{k=1}\beta_k\varphi_k(x)         \\
% 		        & = \sum^{\infty}_{k=1}\beta_k\sqrt{2}\sin(k\pi x)
% 	      \end{align*}
% 	      Die Koeffizienten berechnet man durch
% 	      \begin{align*}
% 		      \beta_k & =\int^1_0 a\sqrt{2}\sin(k\pi x)dx                                         \\
% 		              & = a \sqrt{2} \left[ \left. -\frac{\cos(k \pi x)}{k\pi}\right|^1_0 \right] \\
% 		              & =\frac{a\sqrt{2}}{k\pi}(1-\cos(k\pi))=\begin{cases}\frac{2a\sqrt{2}}{k\pi} & , k\text{ ungerade} \\
% 			      0                       & , \text{sonst.}
% 		      \end{cases}
% 	      \end{align*}
% 	      Der Quellterm hat offenbar die Entwicklung
% 	      \begin{align*}
% 		                         & \sin(\pi x) = \sum_{k=1}^{\infty}\gamma_k \sqrt{2} \sin(k \pi x) \\
% 		      \mathrm{mit} \quad & \gamma_k = \begin{cases}\frac{1}{\sqrt{2}} & , k = 1         \\
% 			      0                  & , \text{sonst.}
% 		      \end{cases}
% 	      \end{align*}

% 	\item F"ur die L"osung $u$ wird der Eigenfunktionen-Ansatz
% 	      \begin{align*}
% 		      u(x,t)=\sum^{\infty}_{k=1}\alpha_k(t)\varphi_k(x)
% 	      \end{align*}
% 	      gew"ahlt, mit zeitabh"angigen Koeffizienten $\alpha_k(t), \, t>0, k\in \mathbb{N}$. \\
% 	      Einsetzen der Entwicklungen von $u(x,t)$ und $\sin(\pi x)$ in die Differentialgleichung liefert
% 	      \begin{align*}
% 		                           & (\sum^{\infty}_{k=1}\alpha_k(t)\varphi_k(x))_t=(\sum^{\infty}_{k=1}\alpha_k(t)\varphi_k(x))_{xx}+\sum^{\infty}_{k=1}\gamma_k\varphi_k(x) \\
% 		      \Leftrightarrow\quad & \sum^{\infty}_{k=1}\alpha'_k(t)\varphi_k(x)=-\sum^{\infty}_{k=1}\alpha_k(t)\lambda_k\varphi_k(x)+\sum^{\infty}_{k=1}\gamma_k\varphi_k(x) \\
% 		      \Leftrightarrow\quad & \sum^{\infty}_{k=1}(\alpha'_k(t)+\lambda_k\alpha_k(t)-\gamma_k)\varphi_k(x)=0
% 	      \end{align*}
% 	      Wir erhalten eine gew"ohnliche Differentialgleichung f"ur die Koeffizienten $\alpha_k(t)$
% 	      \begin{align*}
% 		      \alpha'_k(t) & =-\lambda_k\alpha_k(t)+\gamma_k
% 	      \end{align*}
% 	      mit der allgemeinen L"osung 
% 	      \begin{align*}
% 		      \alpha_k(t) & =\frac{\gamma_k}{\lambda_k}+C_k e^{-\lambda_k t}.
% 	      \end{align*}
% 	      Die Konstante $C_k$ wird durch die Anfangsbedingung f"ur $\alpha_k(0)$ bestimmt. Dazu betrachten wir die Anfangsbedingung f"ur $u$:
% 	      \begin{align*}
% 		      u(x,0)=\sum^{\infty}_{k=1}\alpha_k(0)\sqrt{2}\sin(k\pi x)\stackrel{!}{=} a = \sum_{k=0}^{\infty}\frac{2\sqrt{2}a}{(2k+1)\pi}\sqrt{2}\sin((2k+1)\pi x) \\
% 		      \Rightarrow \alpha_{2j}(0) = 0 \quad \mathrm{und} \quad \alpha_{2j+1}(0) = \frac{2\sqrt{2}a}{(2j+1)\pi} \forall\ j 
% 	      \end{align*}
% 	      Also gilt f"ur $j=0$
% 	      \begin{align*}
% 		      \alpha_1(0)     & = \frac{\gamma_1}{\lambda_1}+C_1  \stackrel{!}{=}\frac{2\sqrt{2}a}{\pi}                               \\
% 		      \Rightarrow C_1 & =\frac{2\sqrt{2}a}{\pi} -\frac{\gamma_1}{\lambda_1} = \frac{2\sqrt{2}a}{\pi}-\frac{1}{\pi^2 \sqrt{2}}
% 	      \end{align*}
% 	      und f"ur $j>0$
% 	      \begin{align*}
% 		      \alpha_{2j}(0)       & =\frac{\gamma_{2j}}{\lambda_{2j}}+C_{2j} = C_{2j} \stackrel{!}{=}0                                    \\
% 		      \Rightarrow C_{2j}   & =0                                                                                                    \\
% 		      \alpha_{2j+1}(0)     & =\frac{\gamma_{2j+1}}{\lambda_{2j+1}}+C_{2j+1} = C_{2j+1} \stackrel{!}{=}\frac{2\sqrt{2}a}{(2j+1)\pi} \\
% 		      \Rightarrow C_{2j+1} & =\frac{2\sqrt{2}a}{(2j+1)\pi}
% 	      \end{align*}
% 	      Die L"osung $u$ lautet daher insgesamt 
% 	      \begin{align*}
% 		      u(x,t) & =\left(\frac{\gamma_1}{\lambda_1} + \left(\frac{2\sqrt{2}a}{\pi}-\frac{1}{\pi^2 \sqrt{2}}\right)e^{-\lambda_1 t}\right)\sqrt{2}\sin(\pi x) + \sum^{\infty}_{k=2} C_ke^{-\lambda_k t}\sqrt{2}\sin(k\pi x) \\
% 		             & = \left(\frac{4a}{\pi} - \frac{1}{\pi^2}\right)e^{-\pi^2 t} \sin(\pi x) + \frac{1}{\pi^2} \sin(\pi x)                                                                                                    \\
% 		             & + \sum_{j=1}^{\infty}\frac{4a}{(2j+1)\pi}e^{-(2j+1)^2 \pi^2 t} \sin((2j+1)\pi x)
% 	      \end{align*}
% \end{enumerate}

% %------------------------------------------------------------------------------
% \newpage
% \begin{example}
% 	Examine the following problem
% 	\begin{align*}
% 		\partial_{tt} u(x,t) & =   \partial_{xx}u(x,t) + a, &  & x\ \in\ (0,1), t>0, \\
% 		u(0,t)               & =0,                          &  & t>0,                \\
% 		u(1,t)               & =0,                          &  & t>0,                \\
% 		u(x,0)               & = x,                         &  & x\, \in\, (0,1),    \\
% 		u_t(x,0)             & = 0,                         &  & x\, \in\, (0,1),
% 	\end{align*}
% 	with constant source term $a\in\mathbb{R}$.
% 	\begin{enumerate}
% 		\item Compute the Eigenvalues and the normalized Eigenfunctions from $\partial_{xx}$.
% 		\item Develop the IC and source term in these Eigenfunctions.
% 		\item Compute the solution of the problem with the Eigenfunction ansatz.
% 	\end{enumerate}
% \end{example}

% \newpage
% (Example 4 cont.)

% Die Gleichung $y''(x) = -cy(x) +b$ besitzt die allg. Lösung 
% 	\begin{align*}
% 		y(x) = \frac{b}{c} + k_1 \sin(\sqrt{c} x)+ k_2\cos(\sqrt{c} x), \quad k_1,k_2 \in \mathbb{R}
% 	\end{align*}

% \begin{enumerate}
% 	\item Die zum Problem geh"orenden Eigenfunktionen und Eigenwerte erf"ullen
% 	      \begin{align*}
% 		      \varphi^{\prime\prime}(x)+\lambda\varphi(x) & =0, \quad x\in (0,1) \\
% 		      \varphi(0)=\varphi(1)                       & =0
% 	      \end{align*}
% 	      \begin{align*}
% 		      \Rightarrow \quad \varphi(x) = C_1\sin(\sqrt{\lambda}x) + C_2 \cos(\sqrt{\lambda}x)
% 	      \end{align*}
% 	      Mit den Randbedingungen f"ur $\varphi$ erh"alt man
% 	      \begin{align*}
% 		      0 = \varphi(0) & = C_2 \, \Rightarrow \, C_2 = 0                                                       \\
% 		      0 = \varphi(1) & = C_1\sin(\sqrt{\lambda}x) \, \Rightarrow \, \lambda_k = (k\pi)^2, \, k\in \mathbb{N}
% 	      \end{align*}
% 	      \begin{align*}
% 		      \Rightarrow \varphi_k(x) = C_1\sin(k\pi x) , \, k \in \mathbb{N}
% 	      \end{align*}
% 	      $C_1$ is obtained by taking into the consideration of normalization
% 	      \begin{align*}
% 		      1 & \stackrel{!}{=} \int_0^1 \varphi_k(x)^2\, dx = C_1^2 \int_0^1 \sin^2(k\pi x) \, dx                \\
% 		        & = C_1^2 \left[ \left. \frac x 2 - \frac{\sin(2k\pi x)}{4k\pi}\right|^1_0 \right] = C_1^2\frac 1 2 \\
% 		      \Rightarrow C_1 = \sqrt{2}
% 	      \end{align*}
% 	      \reversemarginpar
% 	      damit lauten die Eigenwerte und normierten Eigenfunktionen
% 	      \begin{align*}
% 		      \varphi_k(x)               & =\sqrt{2}\sin(k\pi x)                  \\
% 		      \text{und} \quad \lambda_k & =(k\pi)^2 \text{ f"ur } k\in\mathbb{N}
% 	      \end{align*}

% 	\item Die Entwicklung des konstanten Quellterms $a(x,t)=a$  in Eigenfunktionen lautet
% 	      \begin{align*}
% 		      a & = \sum^{\infty}_{k=1}\beta_k\varphi_k(x)         \\
% 		        & = \sum^{\infty}_{k=1}\beta_k\sqrt{2}\sin(k\pi x)
% 	      \end{align*}
% 	      Die Koeffizienten berechnet man durch
% 	      \begin{align*}
% 		      \beta_k & =\int^1_0 a\sqrt{2}\sin(k\pi x)dx                                         \\
% 		              & = a \sqrt{2} \left[ \left. -\frac{\cos(k \pi x)}{k\pi}\right|^1_0 \right] \\
% 		              & =\frac{a\sqrt{2}}{k\pi}(1-\cos(k\pi))=\begin{cases}\frac{2a\sqrt{2}}{k\pi} & , k\text{ ungerade} \\
% 			      0                       & , \text{sonst}
% 		      \end{cases}
% 	      \end{align*}
% 	      Für die Entwicklung der Anfangsbedingung $u(x,0) = x$ gilt:
% 	      \begin{align*}
% 		      x & = \sum^{\infty}_{k=1}\gamma_k\varphi_k(x)         \\
% 		        & = \sum^{\infty}_{k=1}\gamma_k\sqrt{2}\sin(k\pi x)
% 	      \end{align*}
% 	      mit
% 	      \begin{align*}
% 		      \gamma_k & = \int_{0}^{1} x \sqrt{2} \sin(k\pi x)dx                                                                               \\
% 		               & = \sqrt{2} \left[ - x \cos(k \pi x) \frac{1}{k \pi}  \right]_0^1 + \sqrt{2}  \int_0^1 \frac{1}{k \pi} \cos(k \pi x) dx \\
% 		               & =\frac{\sqrt{2}}{k \pi}  \cos(k \pi)+ \frac{\sqrt{2}}{k \pi} \left[ \frac{1}{k \pi} \sin(k \pi x) \right]_0^1          \\
% 		               & = (-1)^k \frac{\sqrt{2}}{k \pi}
% 	      \end{align*}

% 	\item F"ur die L"osung $u$ wird der Eigenfunktionen-Ansatz
% 	      \begin{align*}
% 		      u(x,t)=\sum^{\infty}_{k=1}\alpha_k(t)\varphi_k(x)
% 	      \end{align*}
% 	      gew"ahlt, mit zeitabh"angigen Koeffizienten $\alpha_k(t), \, t>0, k\in \mathbb{N}$. \\
% 	      Einsetzen der Entwicklungen von $u(x,t)$ und $a$ in die Differentialgleichung liefert
% 	      \begin{align*}
% 		                           & (\sum^{\infty}_{k=1}\alpha_k(t)\varphi_k(x))_{tt}=  (\sum^{\infty}_{k=1}\alpha_k(t)\varphi_k(x))_{xx}+\sum^{\infty}_{k=1}\beta_k\varphi_k(x) \\
% 		      \Leftrightarrow\quad & \sum^{\infty}_{k=1}\alpha''_k(t)\varphi_k(x)=-\sum^{\infty}_{k=1}\alpha_k(t)\lambda_k\varphi_k(x)+\sum^{\infty}_{k=1}\beta_k\varphi_k(x)     \\
% 		      \Leftrightarrow\quad & \sum^{\infty}_{k=1}(\alpha''_k(t)+\lambda_k\alpha_k(t)-\beta_k)\varphi_k(x)=0
% 	      \end{align*}
% 	      \normalmarginpar
% 	      Wir erhalten eine gew"ohnliche Differentialgleichung f"ur die Koeffizienten $\alpha_k(t)$
% 	      \begin{align*}
% 		      \alpha''_k(t) & =-\lambda_k\alpha_k(t)+\beta_k
% 	      \end{align*}
% 	      mit der allgemeinen L"osung (Hinweis)
% 	      \begin{align*}
% 		      \alpha_k(t) & =\frac{\beta_k }{\lambda_k}+ C_{k,1} \sin(\sqrt{\lambda_k} t) + C_{k,2} \cos(\sqrt{\lambda_k} t) .
% 	      \end{align*}
% 	      Die Konstanten werden durch die Anfangsbedingungen bestimmt. Dazu gilt zunächst wegen $u_t(x,0) = 0$:
% 	      \begin{align*}
% 		      u_t(x,0) = \sum_{k = 1}^{\infty} \alpha'_k(0) \sqrt{2} \sin(k\pi x) \stackrel{!}{=} 0                            \\
% 		      \Rightarrow \alpha'_k(0) = \sqrt{\lambda_k} C_{k,1} \cos(0) - \sqrt{\lambda_k} C_{k,2} \sin(0) \stackrel{!}{=} 0 \\
% 		      \Rightarrow C_{k,1} = 0
% 	      \end{align*}
% 	      Die Konstante $C_{k,2}$ wird durch die Anfangsbedingung $u(x,0) = x$ bestimmt. Dazu betrachte
% 	      \begin{align*}
% 		      u(x,0)=\sum^{\infty}_{k=1}\alpha_k(0)\sqrt{2}\sin(k\pi x)\stackrel{!}{=} \sum^{\infty}_{k=1}\gamma_k\sqrt{2}\sin(k\pi x) \\
% 		      \Rightarrow \alpha_k(0)\stackrel{!}{=} \gamma_k
% 	      \end{align*}
% 	      Also gilt
% 	      \begin{align*}
% 		      \alpha_k(0)         & =  \frac{\beta_k}{\lambda_k}+C_{k,2} \stackrel{!}{=} \gamma_k =(-1)^k \frac{\sqrt{2}}{k \pi} \\
% 		      \Rightarrow C_{k,2} & = \begin{cases} - \frac{2 a \sqrt{2} }{(k \pi)^3} - \frac{\sqrt{2}}{ k \pi}, k \text{ ungerade} \\
% 			      \frac{\sqrt{2}}{ k \pi},  \text{ sonst}
% 		      \end{cases}
% 	      \end{align*}
% 	      Die L"osung $u$ lautet daher insgesamt
% 	      \begin{align*}
% 		      u(x,t) & = \sum_{k=1}^{\infty} \left( \frac{\beta_k }{\lambda_k}+ C_{k,1} \sin(\sqrt{\lambda_k} t) + C_{k,2} \cos(\sqrt{\lambda_k} t) \right) \sqrt{2} \sin(k \pi x)                        \\
% 		             & = \sum_{k=1 \atop k \text{ gerade}}^{\infty} \left[  \frac{4a }{(k \pi)^3}  +\left( \frac{{2}}{ k \pi} \right)   \cos(k \pi t)  \right]  \sin(k \pi x)                             \\
% 		             & +  \sum_{k=1 \atop k \text{ ungerade}}^{\infty} \left[  \frac{4a }{(k \pi)^3}  - \left( \frac{{2}a }{(k \pi)^3} + \frac{{2}}{ k \pi} \right) \cos(k \pi t)  \right]  \sin(k \pi x)
% 	      \end{align*}
% \end{enumerate}

% \section{Conservation laws - Idea}
% % Integral form
% % \begin{equation}
% % 	\frac{d}{dt}\int_{\Omega}u\,d\Omega 
% % 	= -\int_{\partial\Omega}f(u)\cdot n\,d\partial\Omega
% % \end{equation}
% % Differential form
% % \begin{equation}
% % 	u_{t} + \nabla\cdot f(u) = 0 
% % \end{equation}
% \emph{Master balance principle}: due to the fact that the mathematical structure of the fundamental balance relations,
% namely of mass, linear momentum, moment of momentum, energy and entropy, is in principle identical, they can be 
% formulated within the concise shape of a master balance. To begin with, let $\Psi$ and $\BPsi$ be volume specific
% scalar and vector value densities of a physical quantity to be balanced, respectively. Then, the general balance relations take 
% the global \textbf{integral form} as follows 
% \begin{align}
% 	\frac{d}{dt}\int_{\Omega} \Psi \,d\Omega  & = 
% 	-\int_{\partial\Omega}\Bphi\cdot\Bn\,d\partial\Omega 
% 	+ \int_{\Omega}\sigma\,d\Omega
% 	+ \int_{\Omega}\hat{\Psi}\,d\Omega             \\
% 	\frac{d}{dt}\int_{\Omega} \BPsi \,d\Omega & = 
% 	-\int_{\partial\Omega}\BPhi\cdot\Bn\,d\partial\Omega 
% 	+ \int_{\Omega}\Bsigma\,d\Omega
% 	+ \int_{\Omega}\hat{\BPsi}\,d\Omega
% \end{align}
% where $(\Bphi\cdot\Bn)$ and $(\BPhi\cdot\Bn)$ describe action at the vicinity; $\sigma$ and 
% $\Bsigma$ present for action from a distance; and $\hat{\Psi}$ and $\hat{\BPsi}$ for production or nucleation.
% \begin{example}
% 	Conservation of mass: A derivation from integral to differential form.
% \end{example}
% \inputfig{floats/massx1x2}{massx1x2}
% Let $x$ represent the distance along the tube and let $\rho(x,t)$ be the mass density of the 
% fluid at point $x$ and time $t$. Then, the total mass $M$ in the section 
% $[x_{1},x_{2}]$ at time $t$ is defined as follows
% \begin{equation}
% 	M := \int_{x_{1}}^{x_{2}} \rho(x,t)\,dx.
% \end{equation}
% Assume that the walls of the tube are impermeable and mass is neither created nor destroyed, 
% this total mass $M$ in this interval $[x_1,x_2]$ varies over time only due to the fluid flowing
% across at the boundaries $x=x_{1}$ and $x=x_{2}$. 
% % Besides, the flow rate (also called \emph{rate of flow}, or \emph{flux}) is denoted as $f(\rho)$.
% Now let $v(x,t)$ be the measured velocity of flow at point $x$ and time $t$.
% Then, the flow rate (also called \emph{rate of flow}, or \emph{flux})
% passing the end points of the interval $[x_{1},x_{2}]$ at time $t$ is given by
% \begin{equation}
% 	\rho(x_{1},t)v(x_{1},t)-\rho(x_{2},t)v(x_{2},t).
% \end{equation}
% The rate of change of mass in $[x_{1},x_{2}]$ given by the difference of fluxes at 
% $x_1$ and $x_2$ flow rate has to be equal with the time rate of change of total mass
% \begin{equation}\label{eq:relation0}
% 	\boxed{
% 		\frac{d}{dt}\int_{x_{1}}^{x_{2}} \rho(x,t)\,dx 
% 		% \stackrel{!}{=} f(\rho(x_{1},t))-f(\rho(x_{2},t))
% 		= \rho(x_{1},t)v(x_{1},t)-\rho(x_{2},t)v(x_{2},t)
% 	}
% \end{equation}
% which is the \textbf{integral form} of the conservation law. Next, note that the following relation holds
% \begin{equation}\label{eq:relation1}
% 	\int_{x_{1}}^{x_{2}} \frac{\partial}{\partial x}f(\rho)\,dx 
% 	= f(\rho(x_{2},t))-f(\rho(x_{1},t)).
% \end{equation}
% Substitution of \eqref{eq:relation1} into \eqref{eq:relation0} leads to
% \begin{equation}\label{eq:relation2}
% 	\frac{d}{dt}\int_{x_{1}}^{x_{2}} \rho(x,t)\,dx 
% 	= -\int_{x_{1}}^{x_{2}}\frac{\partial}{\partial x}f(\rho)\,dx.
% \end{equation}
% If $\rho(x,t)$ is sufficiently smooth, time derivative on the LHS of \eqref{eq:relation2}
% can be brought inside the integral. Then, an arrangement of \eqref{eq:relation2} yields
% \begin{equation}\label{eq:relation3}
% 	\int_{x_{1}}^{x_{2}}
% 	\left(\frac{\partial}{\partial t}\rho(x,t) + \frac{\partial}{\partial x}f(\rho)\right)\,dx 
% 	= 0.
% \end{equation}
% Since \eqref{eq:relation3} holds for any choice of $x_{1}$ and $x_{2}$, the integrant must be vanished
% everywhere. Therefore, one obtains the \textbf{differential form} of the conservation law
% \begin{equation}\label{eq:deri}
% 	\therefore \quad
% 	\boxed{\rho_{t} + f(\rho)_{x} = 0}
% \end{equation}
% % \begin{enumerate}
% % 	\item gas dynamics
% % 	\item oil-water mixtures
% % 	\item plasmas
% % 	\item shallow water flows
% % 	\item meteology
% % 	\item traffic
% % 	\item ...
% % \end{enumerate}
% % \begin{definition}
% % 	Cauchy problem
% % \end{definition}
% % Cauchy problem\\
% % Riemann problem\\
% % \begin{equation}
% % 	u_{t} + f(u)_{x} = 0 
% % \end{equation}
% \begin{example}
% 	Different definitions of flux function yields different applications
% \end{example}
% \begin{enumerate}
% 	\item Advection equation: \eqref{eq:deri} with flux function
% 	      $\displaystyle  f(\rho) =  a\rho, \ a\in \mathbb{R^+}$, taking form
% 	      \begin{equation}
% 		      \boxed{
% 			      \rho_{t} + a\rho_{x} = 0
% 		      }
% 	      \end{equation}
% 	\item \emph{Inviscid} Burgers' equation: \eqref{eq:deri} with flux function
% 	      $\displaystyle  f(\rho) =  \frac{1}{2}\rho^2$, yielding
% 	      \begin{equation}
% 		      \boxed{
% 			      \rho_{t} + \rho\rho_{x} = 0
% 		      }
% 	      \end{equation}
% 	      which is used to illustrate the distortion of waveform in simple waves.
% 	      Meanwhile, \emph{viscous} Burgers' equation is a scalar parabolic PDE, taking the form
% 	      $$\rho_t + \rho\rho_{x} = \varepsilon\rho_{xx}$$
% 	      which is the simplest differential model for a fluid flow.
% 	\item Lighthill-Whitham-Richards (LWR) equation: \eqref{eq:deri} with flux function
% 	      $$\displaystyle  f(\rho) =  u_{max}\,\rho\left(1-\frac{\rho}{\rho_{max}}\right)$$
% 	      which is used to model traffic flow; $u_{max}$ is the given maximal speed of vehicle to be allowed, 
% 	      $\rho_{max}$ the given maximal density of vehicle; taking the form
% 	      \begin{equation}
% 		      \rho_{t} + \left(u_{max}-\frac{2u_{max}}{\rho_{max}}\rho\right)\rho_{x} = 0
% 	      \end{equation}
% 	\item (Typical) Buckley-Leverett petroleum equation: \eqref{eq:deri} with flux function
% 	      $$\displaystyle f(\rho) =  \frac{\rho^2}{\rho^2 + (1-\rho)^2\mu_{\text{water}}/\mu_{\text{oil}}}$$
% 	      which is a one-dimensional model for a two-phase flow; $\rho$ stands for water saturation
% 	      and takes the value in the interval $[0,1]$.
% 	\item \emph{Euler} equations of gas dynamics in 1D
% 	      %   \begin{align}
% 	      %       \begin{cases}
% 	      % 	      \begin{aligned}
% 	      % 		      \rho_{t}     & + (\rho v)_x & = 0 \\
% 	      % 		      (\rho v)_{t} & + (\rho v^2 + p)_x & = 0 \\
% 	      % 		      E_{t}        & + (v(E+p))_x & = 0
% 	      % 	      \end{aligned}
% 	      %       \end{cases}
% 	      %   \end{align}
% 	      \begin{align}
% 		      \begin{pmatrix} \rho\\\rho v \\ E \end{pmatrix}_{t}
% 		      +
% 		      \begin{pmatrix} \rho v\\\rho v^2 + p \\ v(E+p) \end{pmatrix}_{x}
% 		      =
% 		      \begin{pmatrix} 0\\ 0 \\ 0 \end{pmatrix}
% 	      \end{align}
% 	      $\rightarrow$ The pressure $p$ should be specified as a given function of 
% 	      mass density $\rho$, linear momentum $\rho v$ and/or energy $E$ in order to 
% 	      fulfill the \emph{closure} problem.
% 	      Such additional equation is called \emph{equation of state} (normally in fluid), or 
% 	      \emph{constitutive equation} (normally in solid). 
% 	\item \emph{Euler} equations of gas dynamics in 2D
% 	      \begin{align}
% 		      \begin{pmatrix} \rho\\ \rho u\\ \rho v \\ E \end{pmatrix}_{t}
% 		      +
% 		      \begin{pmatrix} \rho u\\ \rho u^2 + p \\ \rho uv \\ u(E+p) \end{pmatrix}_{x}
% 		      + 
% 		      \begin{pmatrix} \rho v\\ \rho uv \\ \rho v^2 + p \\ v(E+p) \end{pmatrix}_{y}
% 		      =
% 		      \begin{pmatrix} 0 \\ 0 \\ 0 \\ 0 \end{pmatrix}
% 	      \end{align}
% \end{enumerate}
% % \begin{recall}
% % 	Convexity
% % \end{recall}
% \begin{definition}
% 	Cauchy problem and Riemann problem
% \end{definition}
% Let $u$ be a conserved unknown quantity to be modelled and defined as follows
% \begin{align}\label{eq:conservedquantity}
% 	u:
% 	\begin{cases}
% 		\mathbb{R}\times\mathbb{R}^+ \rightarrow \mathbb{R}, \\
% 		(x,t) \mapsto u(x,t).
% 	\end{cases}
% \end{align}
% \emph{Cauchy's problem} is simply the pure initial value problem (IVP), e.g. 
% find a function $u$ holding \eqref{eq:conservedquantity} that is a solution of
% $\eqref{eq:cauchy}_1$ satisfying the initial condition $\eqref{eq:cauchy}_2$:
% \begin{equation}\label{eq:cauchy}
% 	\boxed{
% 		\begin{aligned}
% 			\partial_{t}u + \partial_{x}f(u) & = 0,        \\
% 			u(x,0)                           & = u_{0}(x).
% 		\end{aligned}
% 	}
% \end{equation}
% \emph{Riemann's problem} is simply the conservation law together with particular initial data
% consisting of two constant states separated by a single discontinuity, e.g.
% find a function $u$ holding \eqref{eq:conservedquantity} that is a solution of
% $\eqref{eq:riemann}_1$ satisfying the initial condition $\eqref{eq:riemann}_2$:
% \begin{equation}\label{eq:riemann}
% 	\boxed{
% 		\begin{aligned}
% 			\partial_{t}u + \partial_{x}f(u) & = 0,        \\
% 			u(x,0)                           & = u_{0}(x)=
% 			\begin{cases}
% 				u_{L}, \quad x<0, \\
% 				u_{R}, \quad x>0.
% 			\end{cases}
% 		\end{aligned}
% 	}
% \end{equation}
% % \begin{definition}
% % 	Hyperbolic
% % \end{definition}
% % Let $\Omega$ be 
% % \inputfig{floats/comparehyperbolic}{comparehyperbolic}
% %------------------------------------------------------------------------------
% \section{Characteristics}
% Let $f:\mathbb{R}\rightarrow\mathbb{R}$ be a $C^1$ function. We consider the \emph{Cauchy} problem
% \begin{align}\label{eq:charac}
% 	\frac{\partial u}{\partial t} + \frac{\partial}{\partial x}f(u) & = 0,        \qquad  x\in\mathbb{R}, t>0, \\
% 	u(x,0)                                                          & = u_{0}(x), \qquad  x\in\mathbb{R}.
% \end{align}
% By setting $a(u) = f'(u)$, and letting $u$ be a classical solution of \eqref{eq:charac},
% one obtains from \eqref{eq:charac} the \emph{non-conservative} form as follows
% \begin{equation}
% 	\frac{\partial u}{\partial t} + a(u)\frac{\partial u}{\partial x} = 0.
% \end{equation}
% The characteristic curves associated with \eqref{eq:charac} are defined as the integral curves
% of the differential equation
% \begin{equation}
% 	\frac{dx}{dt} = a(u(x(t),t)).
% \end{equation}
% \begin{proposition}
% 	Assume that u is a smooth solution of the Cauchy problem. The characteristic curves 
% 	are straight lines along which u is constant.
% \end{proposition}
% \begin{proof}
% 	Consider a characteristic curve passing through the point $(x_{0},0)$, i.e.
% 	a solution of the ordinary differential system
% 	\begin{align}
% 		\begin{cases}\displaystyle
% 			\frac{dx}{dt} & = a(u(x(t),t)), \\
% 			x(0)          & = x_{0}.
% 		\end{cases}
% 	\end{align}
% 	It exists at least on a small time interval $[0,t_{0})$. 
% 	Along such a curve, u is constant since
% 	\begin{align*}
% 		\frac{d}{dt}u(x(t),t) 
% 		 & = \frac{\partial u}{\partial t}(x(t),t)
% 		+ \frac{\partial u}{\partial x}(x(t),t) \frac{dx}{dt}(t)                                    \\
% 		 & =\left(\frac{\partial u}{\partial t} + a(u)\frac{\partial u}{\partial x}\right)(x(t),t) 
% 		= 0.
% 	\end{align*}
% 	Therefore, the characteristic curves are straight lines whose constant slopes depend
% 	on the initial data. As a result, the characteristic straight line passing through
% 	the point $(x_{0},0)$ is defined by the equation
% 	\begin{equation}
% 		\boxed{
% 			x = x_{0} + ta(u_{0}(x_{0}))
% 		}
% 	\end{equation}
% \end{proof}
% \begin{example}
% 	Determine characteristics of the inviscid Burgers' equation with initial conditions given as
% 	\begin{equation}
% 		u(x,0) = u_{0}(x) =
% 		\begin{cases}
% 			\begin{aligned}
% 				 & 1,   & x\leq 0,        &                        \\
% 				 & 1-x, & 0\leq x \leq 1, & \quad x\in \mathbb{R}, \\
% 				 & 0,   & x\geq 1.        & 
% 			\end{aligned}
% 		\end{cases}
% 	\end{equation}
% \end{example}
% By using the method of characteristics, the solution can be solved up to the time when
% those characteristics first intersect with each other, i.e. so-called breaking time or shock.
% Since the $f'(u) = u$ in the inviscid Burgers' equation, it yields the characteristics 
% passing through the point $(x_{0},t)$
% \begin{equation}
% 	x = x_{0} + t\,u_{0}(x_{0})
% \end{equation}
% which leads to
% \begin{equation}
% 	x(x_{0},t) = 
% 	\begin{cases}
% 		\begin{aligned}
% 			 & x_{0}+t,          & x_{0}\leq 0,          \\
% 			 & x_{0}+t(1-x_{0}), & 0\leq x_{0} \leq 1,   \\
% 			 & x_{0},            & x_{0}\geq 1.         
% 		\end{aligned}
% 	\end{cases}
% \end{equation}
% (Sketch)
% %------------------------------------------------------------------------------
% % \section{General strategy for solving the first-order PDE}
% % Supposed 
% %------------------------------------------------------------------------------
% % \section{Rankine-Hugoniot (RH) jump condition}

% % Characteristic curves
% % \begin{equation}
% % 	\dot{x} = f'(u)\\
% % \end{equation}
% % \begin{align*}
% % 	\frac{d}{dt}u(x(t),t) 
% % 	= \frac{\partial u}{\partial t}(x(t),t)
% % 	+ \frac{\partial u}{\partial x}(x(t),t) \frac{dx}{dt}(t)
% % \end{align*}


% % \inputfig{floats/multistructures}{multistructures}

\end{document}