\documentclass[12pt]{article}
\usepackage{anysize}
\usepackage[ngerman,english]{babel}
\marginsize{3.5cm}{2.5cm}{1cm}{2cm}

\usepackage[utf8]{inputenc}
\usepackage[english]{babel}
\usepackage{amsmath}
\usepackage{amsthm}
\usepackage{amsfonts}
\usepackage{scalerel,amssymb}  % for mathbb
%---------------------------------------------------------
\usepackage{authblk}
\usepackage{blindtext}
\usepackage{graphicx}
\usepackage[caption=true]{subfig}
\usepackage[
	colorlinks = true,
	citecolor = red,
	% linkcolor = darkblue, % internal references
	% urlcolor = darkblue,
]{hyperref}
\usepackage{cleveref}
\crefname{figure}{Figure}{Figures}
\crefname{equation}{Eq.}{Eqs.}
%---------------------------------------------------------
% \newcommand{\begin{equation}}{\eqbeg} 
% \newcommand{\end{equation}}{\eqend}
\usepackage{bm}
\usepackage{dfcmd}
% \newcommand{\Bvarepsilon}{\bm\varepsilon} 
% \newcommand{\Bx}{\bm x} 
% \newcommand{\Bu}{\bm u} 
% \newcommand{\Bv}{\bm v} 
% \newcommand{\Bg}{\bm g} 
% \newcommand{\Bb}{\bm b} 
\newcommand{\divv}{\text{divv}}
%---------------------------------------------------------
\usepackage{lineno}
% \linenumbers     

\usepackage{float}
\usepackage{psfrag}

\usepackage{fontawesome}
\usepackage{relsize}

%---------------------------------------------------------
%For \toprule \midrule \bottomrule in table environment
\usepackage{booktabs}

%For theorem into a box
\usepackage{mdframed}
% \newmdtheoremenv{theo}{Theorem}
\newmdtheoremenv{interestingfactboxed}{Interesting fact}
\newmdtheoremenv{observationboxed}{Observation}
\newmdtheoremenv{exampleboxed}{Example}
\newmdtheoremenv{questionboxed}{Question}
\newmdtheoremenv{summaryboxed}{Summary}

%For big integral sign
\usepackage{bigints}

%For [i)] in enumerate
\usepackage{enumerate}

%For cancelto
\usepackage{cancel}
%Page numbering in style 1/3...
\usepackage{lastpage}  
\usepackage{hyperref}
\makeatletter
\renewcommand{\@oddfoot}{\hfil 
% Aachen, November $04^{th}$, 2021 \hspace{300pt} 
Mathe III $\cdot$ GUE09 $\cdot$ WS22/23 
\hspace{250pt} 
\thepage/\pageref{LastPage}\hfil}
\makeatother
%------------------------------------------------------------------------------
\begin{document}
\begin{center}
	\section*{Global exercise - GUE09}
\end{center}
\begin{center}
	Tuan Vo
\end{center}
\begin{center}
	$15^{\text{th}}$ December, 2022
\end{center}
Content covered:
\begin{itemize}
	\item[\checkmark] Analysis:
		\begin{enumerate}[(i)]
			\item Step-by-step with Line integral of scalar field
			\item Step-by-step with Line integral of vector field
		\end{enumerate}
	\item[\checkmark] Numerics: Demo programming exercise 03 (PRU02)
		%       \begin{enumerate}
		% 	      \item Review: QR-decomposition by using
		% 	            \begin{enumerate}
		% 		            \item Givens-Rotation
		% 		            \item Householder-Reflection
		% 	            \end{enumerate}
		% 	      \item QR algorithm to find all eigenvalues of a matrix $A$.
		%       \end{enumerate}
		% \item[\checkmark] Analysis: Application of Hölder's inequality to approximate integral
\end{itemize}
%------------------------------------------------------------------------------
\begin{recallboxed}
	\label{recall:scalar}
	Line integral of a scalar field $\phi: \Omega \to \mathbb{R}$ is defined as follows
	\begin{align}
		\bigintsss_{\Gamma} \phi \, ds 
		:= \bigintsss_{a}^{b} \phi(\gamma(t)) \big\| \gamma'(t) \big\| \, dt
	\end{align}
\end{recallboxed}
\begin{recallboxed}
	\label{recall:vector}
	Line integral of a vector field $\Bf: \Omega \to \mathbb{R}^{n}$ is defined as follows
	\begin{align}
		\bigintsss_{\Gamma} \Bf \cdot d\Bx 
		:= \bigintsss_{a}^{b} \big\langle \Bf(\gamma(t)), \gamma'(t) \big\rangle \, dt
	\end{align}
\end{recallboxed}
%------------------------------------------------------------------------------
\clearpage
\section{Step-by-step with Line integral of scalar field}
\begin{exampleboxed}
	Examine the following trajectory $\Gamma = \gamma([0,2\pi])$
	\begin{align}
		\gamma:
		\begin{cases}
			[0,2\pi]\rightarrow\mathbb{R}^3 , \\ 
			(t) \mapsto \gamma(t) := 
			\begin{pmatrix} \cos(t) \\ \sin(t) \\ ht \end{pmatrix}.
		\end{cases}
	\end{align}
	The curve $\gamma$ is known as the helix. 
	We now would like to compute the total mass $\mathcal{M}$ of the helix
	whose density function $\rho$ defined as follows
	\begin{align}
		\rho: 
		\begin{cases}
			\Gamma \to \mathbb{R}, \\
			(x,y,z) \mapsto \rho(x,y,z) := z.
		\end{cases}
	\end{align}
\end{exampleboxed}
Approach: By using \cref{recall:scalar}.
\begin{observationboxed}
	Since the parametrized curve $\gamma$ from the given trajectory or path $\Gamma$ is already known, 
	we, therefore, do not have to seek any further parametrized curve.
	However, in case this information is not yet already given, it is necessary to find such parametrized curve,
	i.e. the \cref{section2} is such an example.
\end{observationboxed}
The mass of the helix is computed as follows
\begin{align*}
	\therefore\quad\boxed{
		\begin{aligned}
			\mathcal{M}^{\text{helix}}
			 & = \bigintsss_0^{2\pi} \rho \, {\|\gamma'(t)\|}_2 \, dt               \\
			 & = \bigintsss_0^{2\pi} h  t \, \sqrt{\sin^2 t + \cos^2 t + h^2} \, dt \\
			 & = \bigintsss_0^{2\pi} h  \sqrt{1 + h^2} \, t \, dt                   \\
			 & = h  \sqrt{1+h^2} \, \frac{t^2}2 \bigg|_0^{2\pi}                     \\
			 & = 2 \pi^2 h \sqrt{1+h^2}.		
		\end{aligned}
	}
\end{align*}
%------------------------------------------------------------------------------
\clearpage
\section{Step-by-step with Line integral of vector field}
\label{section2}
\begin{exampleboxed}
	Examine the following trajectory $\Gamma_{1}$ as shown in \cref{lineintegral_vectorfield}
	from the origin point $(0, 0)$ to point $(1, 1)$.
	Besides, the vector field $f$ is given as follows
	\begin{align*}
		\Bf:
		\begin{cases}
			\mathbb{R}^{2} \rightarrow \mathbb{R}^{2}, \\
			(x,y) \mapsto \Bf(x,y)
			=
			\begin{pmatrix}
				x e^y \\
				\sin(x) + y
			\end{pmatrix},
		\end{cases}
	\end{align*}
	We would like now to compute the line integral of the vector field $\Bf$, 
	i.e. Work integral/Arbeitsintegral
	\begin{align}
		\bigintsss \limits_{\Gamma_1} \Bf \cdot d\Bx.
	\end{align}
\end{exampleboxed}
\inputfig{floats/lineintegral_vectorfield}{lineintegral_vectorfield}
Approach: By using \cref{recall:vector}.
\clearpage
\begin{observationboxed}
	There is no parametrized curve for the trajectory $\Gamma_1$.
	Therefore, we shall need to find a parametrized curve for the path $\Gamma_1$.
	Since the path $\Gamma_1$ is a union of two smaller paths $l_1$ and $l_2$,
	we obtain the following relation
	\begin{align*}
		\boxed{
			\Gamma_1 := \texttt{Image}(l_1) \cup \texttt{Image}(l_2)
		}
	\end{align*}
	where the parametrized $l_1$ and $l_2$, and their gradients take the forms
	\begin{align*}
		l_1(t) & := \begin{pmatrix} t \\ 0 \end{pmatrix}
		\text{ for } t \in [0,1]
		\Rightarrow l_1' 
		= \begin{pmatrix} 1 \\ 0 \end{pmatrix}           \\
		l_2(t) & := \begin{pmatrix} 1 \\ t \end{pmatrix}
		\text{ for } t \in [0,1]
		\Rightarrow l_2' 
		= \begin{pmatrix} 0 \\ 1 \end{pmatrix}.
	\end{align*}
	Substitution of $l_1(t)$ and $l_2(t)$ into $\Bf$ leads to the following expressions
	\begin{align*}
		\Bf(l_1(t)) & 
		= \begin{pmatrix} t e^0 \\ \sin(t) + 0 \end{pmatrix}
		= \begin{pmatrix} t \\ \sin(t) \end{pmatrix}, \\
		\Bf(l_2(t)) & 
		= \begin{pmatrix} 1 e^t \\ \sin(1) + t \end{pmatrix}
		= \begin{pmatrix} e^t \\ \sin(1) + t \end{pmatrix}.
	\end{align*}
\end{observationboxed}
The line integral of the vector field is computed as follows
\begin{align*}
	\boxed{
		\begin{aligned}
			\bigintsss_{\Gamma_1} \Bf \cdot d\Bx
			= & \bigintsss_{l_1} \Bf \cdot d\Bx 
			+ \bigintsss_{l_2} \Bf \cdot d\Bx                                    \\
			= & \bigintsss_0^1 \Big\langle f(l_1(t)), l_1'(t) \Big\rangle \, dt 
			+ \bigintsss_0^1 \Big\langle f(l_2(t)), l_2'(t) \Big\rangle \, dt    \\
			= & \bigintsss_0^1 t  \, dt
			+ \bigintsss_0^1 \left( \sin(1) + t \right) \,  dt                   \\
			  & =  \sin(1) + \bigintsss_0^1 2t  \, dt                            \\
			  & =  \sin(1) + \Big[ t^2 \Big]_0^1
		\end{aligned}
	}
\end{align*}
Therefore, we obtain
\begin{align*}
	\therefore\quad\boxed{
		\bigintsss_{\Gamma_1} \Bf \cdot d\Bx =  \sin(1) + 1.
	}
\end{align*}
%------------------------------------------------------------------------------
\clearpage
\section{Demo programming exercise 3 (PRU02)}


\bibliographystyle{acm}
% \bibliography{literature.bib}
\end{document}