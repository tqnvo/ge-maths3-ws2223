\documentclass[12pt]{article}
\usepackage{anysize}
\usepackage[ngerman,english]{babel}
\marginsize{3.5cm}{2.5cm}{1cm}{2cm}

\input{packages_vo.tex}

%Page numbering in style 1/3...
\usepackage{lastpage}  
\usepackage{hyperref}

\makeatletter
\renewcommand{\@oddfoot}{\hfil 
% Aachen, November $04^{th}$, 2021 \hspace{300pt} 
Mathe III $\cdot$ GUE13-extra + SRU13 $\cdot$ WS22/23
\hspace{200pt} 
\thepage/\pageref{LastPage}\hfil}
\makeatother
%------------------------------------------------------------------------------
\begin{document}
\begin{center}
	\section*{Global exercise - GUE13-extra + SRU13}
\end{center}
\begin{center}
	Tuan Vo $+$ Jan Habscheid
\end{center}
\begin{center}
	$27^{\text{th}}$ January, 2023
\end{center}
%------------------------------------------------------------------------------
\section{Step-by-step with A2-SRU12}
\begin{mdframed}
	Examine the surface integral in the Stokes' theorem.
\end{mdframed}

\begin{enumerate}[(i)]
	\item Parametrized surface $S$ given in the cylindrical coordinate
	      \begin{align}
		      \Phi(s,t) = 
		      \begin{pmatrix}
			      s \cos(t) \\ s \sin(t) \\ t
		      \end{pmatrix}
		      \quad 
		      \forall \, s\in [0,1], \, t\in[0,\pi/2].
	      \end{align}
	      \inputfig{floats/surface_S}{surface_S}
	      
	\item \textbf{Outward normal vector} of the surface $S$
	      \begin{align}
		      \label{eq:normalvector}
		      \Bn:=
		      \partial_{s}\Phi(s,t) \times \partial_{t}\Phi(s,t)
		       & =
		      \begin{pmatrix}
			      \cos(t) \\ \sin(t) \\ 0
		      \end{pmatrix}
		      \times
		      \begin{pmatrix}
			      -s\,\sin(t) \\ s\,\cos(t) \\ 1
		      \end{pmatrix} \notag \\
		       & =
		      \begin{vmatrix}
			      i           & j          & k \\
			      \cos(t)     & \sin(t)    & 0 \\
			      -s\,\sin(t) & s\,\cos(t) & 1
		      \end{vmatrix}
		      =
		      \begin{pmatrix}
			      \sin(t) \\ -\cos(t) \\  s
		      \end{pmatrix}
	      \end{align}
	      \begin{observationboxed}
		      Note in passing that the $z$-component of the above normal vector 
		      for the surface $S$ is already positive, which satisfies 
		      the convention mentioned in this exercise
		      for an \textbf{outward} normal vector.
		      It is due to the fact that the cross product
		      $\partial_{s}\Phi(s,t) \times \partial_{t}\Phi(s,t)$
		      is different from 
		      $\partial_{t}\Phi(s,t) \times \partial_{s}\Phi(s,t)$
		      in terms of sign. We can have a quick check as follows
		      \begin{align}
			      \partial_{t}\Phi(s,t) \times \partial_{s}\Phi(s,t)
			       & =
			      \begin{pmatrix}
				      -s\,\sin(t) \\ s\,\cos(t) \\ 1
			      \end{pmatrix}
			      \times
			      \begin{pmatrix}
				      \cos(t) \\ \sin(t) \\ 0
			      \end{pmatrix} \\
			       & =
			      \begin{vmatrix}
				      i           & j          & k  \\
				      -s\,\sin(t) & s\,\cos(t) & 1  \\
				      \cos(t)     & \sin(t)    & 0 
			      \end{vmatrix}
			      =
			      \begin{pmatrix}
				      -\sin(t) \\ \cos(t) \\  -s
			      \end{pmatrix}
			      = -\Bn
		      \end{align}
		      whose $z$-component is obviously negative,
		      and according to the convention mentioned in this exercise
		      this normal vector is \textbf{not} an outward normal vector.
		      Nevertheless, either option is still correct,
		      and we just have to be consistent with the 
		      option that we have chosen. 
		      In the context of this exercise,
		      we will go with the positive $z$-component.
	      \end{observationboxed}
	      
	\item \textbf{Norm} of the outward normal vector
	      \begin{align}
		      \label{eq:norm}
		      \| \Bn \|_2 
		      := 
		      \| \partial_{s}\Phi(s,t) \times \partial_{t}\Phi(s,t) \|_2
		      \stackrel{\eqref{eq:normalvector}}{=}
		      \sqrt{\sin^2(t) + \left( - \cos(t)\right)^2 + s}
		      =
		      \sqrt{1+s}
	      \end{align}
	      
	\item \textbf{Outward unit normal vector}: Normalization of the normal vector
	      \begin{align}
		      \label{eq:unitnormalvector}
		      \bm{\nu} := \frac{\Bn}{\| \Bn \|_2 }
		      \stackrel{\eqref{eq:normalvector},\eqref{eq:norm}}{=}
		      \frac{1}{\sqrt{1+s}}
		      \begin{pmatrix}
			      \sin(t) \\ -\cos(t) \\  s
		      \end{pmatrix}
	      \end{align}
	      
	\item Now, consider a vector field $\Bf$ defined as follows
	      \begin{align}
		      \label{eq:vectorfieldf}
		      \Bf:
		      \begin{cases}
			      \mathbb{R}\times\mathbb{R}\times\mathbb{R} \to \mathbb{R}^3 \\
			      (x,y,z) \mapsto \Bf(x,y,z)
			      =
			      \begin{pmatrix}
				      f_{1}(x,y,z) \\ f_{2}(x,y,z) \\f_{3}(x,y,z)
			      \end{pmatrix}
			      :=
			      \begin{pmatrix}
				      z \\x\\y
			      \end{pmatrix}.
		      \end{cases}
	      \end{align}
	      
	\item The Stokes' theorem requires the equality of
	      a surface integral and a line integral, meaning that 
	      \begin{mdframed}
		      \begin{align}
			      \bigintsss_{S} \left( \nabla \times \Bf \right) \cdot d\bm{\sigma}
			      =
			      \bigointsss_{\partial S} \Bf \cdot d \Bs
		      \end{align}
	      \end{mdframed}
	      
	\item Examine the urface integral:
	      \begin{mdframed}
		      \begin{align}
			      \bigintsss_{S} \left( \nabla \times \Bf \right) \cdot d\bm{\sigma}
		      \end{align}
		      which is also written as follows 
		      \begin{align}
			      \label{eq:surfacsimilar}
			      \bigintsss_{S} \left( \nabla \times \Bf \right) \cdot d\bm{\sigma}
			      = 
			      \bigintsss_{S} \left( \nabla \times \Bf \right) \cdot \bm{\nu} \, d\sigma
			      =
			      \bigintsss_{S} \Big\langle \left( \nabla \times \Bf \right), \bm{\nu} \Big\rangle \, d\sigma
		      \end{align}
	      \end{mdframed}
	      
	      The curl of $\Bf$ is computed straightforwardly as follows
	      \begin{align}
		      \label{eq:curl}
		      \nabla \times \Bf 
		       & =
		      \begin{pmatrix}
			      \partial_x \\ \partial_y \\ \partial_z
		      \end{pmatrix}
		      \times
		      \begin{pmatrix}
			      f_{1}(x,y,z) \\ f_{2}(x,y,z) \\ f_{3}(x,y,z)
		      \end{pmatrix} \notag \\
		       & = 
		      \begin{vmatrix}
			      i            & j            & k            \\
			      \partial_x   & \partial_y   & \partial_z   \\
			      f_{1}(x,y,z) & f_{2}(x,y,z) & f_{3}(x,y,z)
		      \end{vmatrix}
		      =
		      \begin{pmatrix}
			      \partial_yf_3 - \partial_zf_2   \\
			      - \partial_xf_3 + \partial_zf_1 \\
			      \partial_xf_2 - \partial_yf_1 
		      \end{pmatrix}
		      \stackrel{\eqref{eq:vectorfieldf}}{=}
		      \begin{pmatrix}
			      1 \\1\\1
		      \end{pmatrix}
	      \end{align}
	      
	      \begin{observationboxed}
		      The integral $\eqref{eq:surfacsimilar}_{3}$
		      is a surface integral of the first kind, i.e.
		      surface integral of a \textbf{scalar} field.
		      It is due to the fact that 
		      the inner product $(\nabla \times \Bf)$
		      and the outward unit normal vector $\bm{\nu}$
		      results in a \textbf{scalar}. 
	      \end{observationboxed}
	      \begin{recallboxed}
		      Surface integral of a 
		      \textbf{scalar field} $\phi: \Omega \to \mathbb{R}$ is defined as
		      \begin{align}
			      \label{recall:scalarsurface}
			      \bigintsss_{\Gamma} \phi \, dA 
			      := \bigintsss_{B} \phi(\Phi(p,q)) \, \big\| \Phi_p \times \Phi_q \big\| \, d(p,q)
		      \end{align}
	      \end{recallboxed}
	      
	      Therefore, the integral $\eqref{eq:surfacsimilar}_{3}$
	      is computed as follows
	      \begin{align}
		      \bigintsss_{S} \Big\langle \left( \nabla \times \Bf \right), \bm{\nu} \Big\rangle \, d\sigma
		       & 
		      \stackrel{\eqref{eq:curl}}{=}
		      \bigintsss_{S} \Big\langle \begin{pmatrix} 1\\1\\1\end{pmatrix}, \bm{\nu} \Big\rangle \, d\sigma \\
		       & 
		      \stackrel{\eqref{recall:scalarsurface}}{=}
		      \bigintsss_{0}^{1} \!\!\!\!\! \bigintsss_{0}^{\pi/2}
		      \Big\langle \begin{pmatrix} 1\\1\\1\end{pmatrix}, \bm{\nu} \Big\rangle 
		      \big\| \Phi_s \times \Phi_t \big\|_2 \, dt \,ds                                                  \\
		       & 
		      \stackrel{\eqref{eq:normalvector}}{=}
		      \bigintsss_{0}^{1} \!\!\!\!\! \bigintsss_{0}^{\pi/2}
		      \Big\langle \begin{pmatrix} 1\\1\\1\end{pmatrix}, \bm{\nu} \Big\rangle 
		      \big\| \Bn \big\|_2 \, dt \,ds                                                                   \\
		       & 
		      \stackrel{\eqref{eq:unitnormalvector}}{=}
		      \bigintsss_{0}^{1} \!\!\!\!\! \bigintsss_{0}^{\pi/2}
		      \Big\langle \begin{pmatrix} 1\\1\\1\end{pmatrix}, \frac{\Bn}{\| \Bn \|_2} \Big\rangle 
		      \big\| \Bn \big\|_2 \, dt \,ds                                                                   \\
		       & 
		      =
		      \bigintsss_{0}^{1} \!\!\!\!\! \bigintsss_{0}^{\pi/2}
		      \Big\langle \begin{pmatrix} 1\\1\\1\end{pmatrix}, \Bn \Big\rangle  \, dt \,ds                    \\
		       & 
		      \stackrel{\eqref{eq:normalvector}}{=}
		      \bigintsss_{0}^{1} \!\!\!\!\! \bigintsss_{0}^{\pi/2}
		      \Big\langle \begin{pmatrix} 1\\1\\1\end{pmatrix}, 
		      \begin{pmatrix} \sin(t) \\ -\cos(t) \\  s \end{pmatrix}
		      \Big\rangle  \, dt \,ds                                                                          \\
		       & 
		      \stackrel{\eqref{eq:normalvector}}{=}
		      \bigintsss_{0}^{1} \!\!\!\!\! \bigintsss_{0}^{\pi/2}
		      \Big( \sin(t) - \cos(t) + s \Big)
		      \, dt \,ds                                                                                       \\
		       & = \dots                                                                                       \\
		       & = \frac{\pi}{4}
	      \end{align}
	      
	      Consequentially, the normal vector $\Bn$ is needed to be normalized, i.e. 
	      we need the unit normal vector $\bm{\nu}$.
	      However, due to the sense of the surface integral of the first kind,
	      the norm of the normal vector is canceled within integral.
	      Therefore, it was actually not necessary to compute the norm, 
	      and we may compute the integral directly with the non-normalized normal vector $\Bn$.
\end{enumerate}

\bibliographystyle{acm}
% \bibliography{literature_vo.bib}
\end{document}