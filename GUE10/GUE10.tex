\documentclass[12pt]{article}
\usepackage{anysize}
\usepackage[ngerman,english]{babel}
\marginsize{3.5cm}{2.5cm}{1cm}{2cm}

\usepackage[utf8]{inputenc}
\usepackage[english]{babel}
\usepackage{amsmath}
\usepackage{amsthm}
\usepackage{amsfonts}
\usepackage{scalerel,amssymb}  % for mathbb
%---------------------------------------------------------
\usepackage{authblk}
\usepackage{blindtext}
\usepackage{graphicx}
\usepackage[caption=true]{subfig}
\usepackage[
	colorlinks = true,
	citecolor = red,
	% linkcolor = darkblue, % internal references
	% urlcolor = darkblue,
]{hyperref}
\usepackage{cleveref}
\crefname{figure}{Figure}{Figures}
\crefname{equation}{Eq.}{Eqs.}
%---------------------------------------------------------
% \newcommand{\begin{equation}}{\eqbeg} 
% \newcommand{\end{equation}}{\eqend}
\usepackage{bm}
\usepackage{dfcmd}
% \newcommand{\Bvarepsilon}{\bm\varepsilon} 
% \newcommand{\Bx}{\bm x} 
% \newcommand{\Bu}{\bm u} 
% \newcommand{\Bv}{\bm v} 
% \newcommand{\Bg}{\bm g} 
% \newcommand{\Bb}{\bm b} 
\newcommand{\divv}{\text{divv}}
%---------------------------------------------------------
\usepackage{lineno}
% \linenumbers     

\usepackage{float}
\usepackage{psfrag}

\usepackage{fontawesome}
\usepackage{relsize}

%---------------------------------------------------------
%For \toprule \midrule \bottomrule in table environment
\usepackage{booktabs}

%For theorem into a box
\usepackage{mdframed}
% \newmdtheoremenv{theo}{Theorem}
\newmdtheoremenv{interestingfactboxed}{Interesting fact}
\newmdtheoremenv{observationboxed}{Observation}
\newmdtheoremenv{exampleboxed}{Example}
\newmdtheoremenv{questionboxed}{Question}
\newmdtheoremenv{summaryboxed}{Summary}

%For big integral sign
\usepackage{bigints}

%For [i)] in enumerate
\usepackage{enumerate}

%For cancelto
\usepackage{cancel}
%Page numbering in style 1/3...
\usepackage{lastpage}  
\usepackage{hyperref}
\makeatletter
\renewcommand{\@oddfoot}{\hfil 
% Aachen, November $04^{th}$, 2021 \hspace{300pt} 
Mathe III $\cdot$ GUE10 $\cdot$ WS22/23 
\hspace{250pt} 
\thepage/\pageref{LastPage}\hfil}
\makeatother
%------------------------------------------------------------------------------
\begin{document}
\begin{center}
	\section*{Global exercise - GUE10}
\end{center}
\begin{center}
	Tuan Vo
\end{center}
\begin{center}
	$22^{\text{nd}}$ December, 2022
\end{center}
Content covered:
\begin{itemize}
	\item[\checkmark] Analysis: Review H9A3 and H9A4
	\item[\checkmark] Analysis: Orientable surface and surface integral
	\item[\checkmark] Numerics: Demo programming exercise 3 (PRU02)
\end{itemize}
%------------------------------------------------------------------------------
\section{Review H9A3}
 (white board)
%------------------------------------------------------------------------------
\section{Review H9A4}
 (white board)
\clearpage
%------------------------------------------------------------------------------
\section{Analysis: Orientable integral}
\begin{exampleboxed}
	Examine the \textbf{Möbius band} given as follows
	\begin{align}
		\vec{\gamma}:
		\begin{cases}
			(-1,1) \times (0,2\pi) \to \mathbb{R}^3, \\
			(t,\phi) \mapsto 
			\vec{\gamma}(t,\phi) := 
			\begin{pmatrix} \cos(\phi) \\ \sin(\phi) \\ 0 \end{pmatrix}
			+ \frac{t}{2}
			\begin{pmatrix} \cos(\phi) \cos(\phi/2)\\ \sin(\phi) \cos(\phi/2)\\\sin(\phi/2) \end{pmatrix}.
		\end{cases}
	\end{align}
	Show that this surface is \textbf{not orientable}.
\end{exampleboxed}
Approach:
\begin{proof} 
	The normal field is computed as follows
	\begin{align*}
		n(t,\phi) 
		 & = \pm 
		\frac{\partial_t \vec{\gamma} \times \partial_{\phi} \vec{\gamma}}
		{\big\|\partial_t \vec{\gamma} \times \partial_{\phi} \vec{\gamma}\big\|} \\
		 & = \frac{\mp 1}{a}
		\begin{pmatrix}
			\cos(\phi) \sin(\frac{\phi}{2}) - \frac{t}{4} \sin(\phi)(1-\cos(\phi)) \\ 
			\sin(\phi) \sin(\phi/2) + \frac{t}{4} (\sin^2(\phi)+ \cos(\phi))       \\ 
			- \cos(\phi/2) - \frac{t}{4}(1+\cos(\phi))
		\end{pmatrix}
	\end{align*}
	with 
	\begin{align*}
		a = \sqrt{1+t \cos(\phi/2)+\frac{t^2}{16}(3+2 \cos(\phi))}.
	\end{align*}
	We choose the positive sign and consider the position 
	$(1,0,0)^T = \vec{\gamma}(0,0) = \vec{\gamma}(0,2\pi)$:
	\begin{align*}
		\lim_{\phi \to 0} n(0,\phi) 
		= \begin{pmatrix} 0\\0\\-1\end{pmatrix} \neq \lim_{\phi\to 2\pi} n(0,\phi) 
		= \begin{pmatrix} 0\\0\\1\end{pmatrix}.
	\end{align*}
	Since the normal field is not continuous on 
	\begin{align*}
		\gamma([-1,1] \times [0,2\pi])
	\end{align*}
	the surface is \textbf{not orientable}.
\end{proof}
\clearpage
%------------------------------------------------------------------------------
\section{Analysis: Surface integral}
\begin{exampleboxed}
	Examine the lateral surface of the \textbf{Frustum} given as follows
	\begin{align*}
		K =
		\left\{
		\bm{x} \in \mathbb{R}^3,\,
		0 < r \leq R
		\, \Bigg| \,
		0 \leq x_3 < H,\,
		0 \leq x_1^2 + x_2^2 <
		\left(R - \frac{R-r}{H} x_3\right)^2
		\right\}.
	\end{align*}
	where $H$ is the height, and $r$ is the radius.
\end{exampleboxed}
Approach:
By using the cylinder coordination 
$\bm{x} = (r \cos (\phi), r \sin(\phi),z)^T$
we obtain the parametrization of the lateral surface of the frustum,
i.e. truncated cone
\begin{align*}
	\bm{\gamma}: 
	\begin{cases}
		(0,2\pi)\times (0,H) \to \mathbb{R}^3,\\
		(\phi,z) \mapsto \bm{\gamma}(\phi,z) :=
	\begin{pmatrix} \left(R - \frac{R-r}{H}z\right) \cos(\phi) \\
		\left(R - \frac{R-r}{H}z\right) \sin(\phi) \\
		z
	\end{pmatrix}.
	\end{cases}
\end{align*}
Wir erhalten das Normalenfeld
\begin{align*}
	\boxed{
		\begin{aligned}
			\partial_{\phi} \bm{\gamma}(\phi,z)\times \partial_z \bm{\gamma}(\phi,z)
			 & = \begin{pmatrix} -\left(R - \frac{R-r}{H}z\right) \sin(\phi) \\
				     \left(R - \frac{R-r}{H}z\right) \cos(\phi)  \\
				     0
			     \end{pmatrix}
			\times
			\begin{pmatrix}
				-\frac{R-r}{H}\cos(\phi) \\-\frac{R-r}{H} \sin(\phi)\\1
			\end{pmatrix}             \\
			 & = \begin{pmatrix}
				     \left(R- \frac{R-r}{H}z\right)\cos(\phi) \\
				     \left(R- \frac{R-r}{H}z\right)\sin(\phi) \\
				     \left(R- \frac{R-r}{H}z\right)\frac{R-r}{H}
			     \end{pmatrix}
		\end{aligned}
	}
\end{align*}
und die Norm des Normalenfelds wird wie folgt berechnet
\begin{align*}
	\boxed{
		||\partial_{\phi} \bm{\gamma} (\phi,z) \times \partial_z \bm{\gamma} (\phi,z)||
		= \left(R - \frac{R-r}{H}z\right) \sqrt{1 + \frac{(R-r)^2}{H^2}}.
	}
\end{align*}
Somit ergibt sich f\"ur den Fl\"acheninhalt der Mantelfl\"ache
\begin{align*}
	\mathcal{A} & =
	\bigintsss_{0}^H \bigintsss_0^{2 \pi}
	||\partial_{\phi} \bm{\gamma} \times \partial_z \bm{\gamma} || \, d\phi \, dz \\
	            & = 2 \pi \sqrt{1 + \frac{(R-r)^2}{H^2}}
	\bigintsss_{0}^H \left(R - \frac{R-r}{H}z\right) \,dt
	= \pi \sqrt{1 + \frac{(R-r)^2}{H^2}}\frac{H}{R-r} \left(R^2-r^2\right).
\end{align*}
Daher ist der Fl\"acheninhalt der Mantelfl\"ache
\begin{align*}
	\therefore\quad\boxed{
		\mathcal{A} =
		\bigintsss_{0}^H \bigintsss_0^{2 \pi}
		||\partial_{\phi} \bm{\gamma} \times \partial_z \bm{\gamma} || \, d\phi \, dz
		= \pi \sqrt{H^2 + (R-r)^2} (R+r).
	}
\end{align*}
%------------------------------------------------------------------------------
\clearpage
\section{Numerics: Demo programming exercise 3 PRU(02)}
\inputfig{floats/pe03_QR_2}{pe03_QR_2}
\inputfig{floats/pe03_QR_4}{pe03_QR_4}
\inputfig{floats/pe03_QR_2_plots}{pe03_QR_2_plots}

\bibliographystyle{acm}
% \bibliography{literature.bib}
\end{document}