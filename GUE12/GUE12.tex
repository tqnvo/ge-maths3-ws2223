\documentclass[12pt]{article}
\usepackage{anysize}
\usepackage[ngerman,english]{babel}
\marginsize{3.5cm}{2.5cm}{1cm}{2cm}

\usepackage[utf8]{inputenc}
\usepackage[english]{babel}
\usepackage{amsmath}
\usepackage{amsthm}
\usepackage{amsfonts}
\usepackage{scalerel,amssymb}  % for mathbb
%---------------------------------------------------------
\usepackage{authblk}
\usepackage{blindtext}
\usepackage{graphicx}
\usepackage[caption=true]{subfig}
\usepackage[
	colorlinks = true,
	citecolor = red,
	% linkcolor = darkblue, % internal references
	% urlcolor = darkblue,
]{hyperref}
\usepackage{cleveref}
\crefname{figure}{Figure}{Figures}
\crefname{equation}{Eq.}{Eqs.}
%---------------------------------------------------------
% \newcommand{\begin{equation}}{\eqbeg} 
% \newcommand{\end{equation}}{\eqend}
\usepackage{bm}
\usepackage{dfcmd}
% \newcommand{\Bvarepsilon}{\bm\varepsilon} 
% \newcommand{\Bx}{\bm x} 
% \newcommand{\Bu}{\bm u} 
% \newcommand{\Bv}{\bm v} 
% \newcommand{\Bg}{\bm g} 
% \newcommand{\Bb}{\bm b} 
\newcommand{\divv}{\text{divv}}
%---------------------------------------------------------
\usepackage{lineno}
% \linenumbers     

\usepackage{float}
\usepackage{psfrag}

\usepackage{fontawesome}
\usepackage{relsize}

%---------------------------------------------------------
%For \toprule \midrule \bottomrule in table environment
\usepackage{booktabs}

%For theorem into a box
\usepackage{mdframed}
% \newmdtheoremenv{theo}{Theorem}
\newmdtheoremenv{interestingfactboxed}{Interesting fact}
\newmdtheoremenv{observationboxed}{Observation}
\newmdtheoremenv{exampleboxed}{Example}
\newmdtheoremenv{questionboxed}{Question}
\newmdtheoremenv{summaryboxed}{Summary}

%For big integral sign
\usepackage{bigints}

%For [i)] in enumerate
\usepackage{enumerate}

%For cancelto
\usepackage{cancel}
%Page numbering in style 1/3...
\usepackage{lastpage}  
\usepackage{hyperref}
\makeatletter
\renewcommand{\@oddfoot}{\hfil 
% Aachen, November $04^{th}$, 2021 \hspace{300pt} 
Mathe III $\cdot$ GUE12 $\cdot$ WS22/23 
\hspace{250pt} 
\thepage/\pageref{LastPage}\hfil}
\makeatother
%------------------------------------------------------------------------------
\begin{document}
\begin{center}
	\section*{Global exercise - GUE12}
\end{center}
\begin{center}
	Tuan Vo
\end{center}
\begin{center}
	$19^{\text{th}}$ January, 2023
\end{center}
\textbf{\textsc{Summary of content covered:}} Numerics
\begin{itemize}
	\item[\checkmark] Optimization: Line search method $+$ Step-by-step standard Newton method
	\item[\checkmark] Optimization: Demo the $4$th programming exercise PRU03
\end{itemize}
%------------------------------------------------------------------------------
\section{Line search method: Main points}
\begin{mdframed}
	\begin{enumerate}
		\item Line search method is an iteration scheme with the form
		      \begin{align}
			      x^{(k+1)} = x^{(k)} + \alpha_k \, d^{(k)}
		      \end{align}
		      
		\item Typically, for the sake of minimum-searching problem we take
		      \begin{align}
			      \nabla f \Big( x^{(k)} \Big) \, d^{(k)} < 0.
		      \end{align}
		      
		\item Searching direction  $d^{(k)}$ takes the form
		      \begin{align}
			      d^{(k)} = - B^{-1}_k \nabla f \Big( x^{(k)} \Big)
		      \end{align}
		      
		      \begin{enumerate}[(i)]
			      \item When $B_k = I$: \textbf{Gradient descent} or
			            \textbf{Steepest descent} method.
			            \begin{align}
				            \begin{cases}
					            x^{(k+1)} = x^{(k)} + \alpha_k \, d^{(k)} \\
					            d^{(k)} = - \nabla f \Big( x^{(k)} \Big)
				            \end{cases}
			            \end{align}
			            
			      \item When $B_k = \nabla^2 f \Big( x^{(k)} \Big)$: \textbf{Newton} method.
			            \begin{align}
				            \begin{cases}
					            x^{(k+1)} = x^{(k)} + \alpha_k \, d^{(k)} \\
					            d^{(k)} = - \left( \nabla^2 f \Big( x^{(k)} \right)^{-1} \nabla f \Big( x^{(k)} \Big)
				            \end{cases}
			            \end{align}
			            
			      \item When $B_k \approx \nabla^2 f \Big( x^{(k)} \Big)$: Quasi-\textbf{Newton} method.
		      \end{enumerate}
		      
		\item Controlling the coefficient $a_k$ leads to examining the two \textbf{Wolfe} conditions:
		      \begin{enumerate}[(i)]
			      \item Upper bound of $a_k$: \textbf{Armijo} condition.
			      \item Lower bound of $a_k$: Curvature condition.
		      \end{enumerate}
	\end{enumerate}
\end{mdframed}

%------------------------------------------------------------------------------
\clearpage
\section{Step-by-step: standard Newton method}
\begin{exampleboxed}
	Given the following function $f$ as follows
	\begin{align}
		f:
		\begin{cases}
			\mathbb{R} \to \mathbb{R}, \\
			x \mapsto f(x):=\sqrt{1+x^2},
		\end{cases}
	\end{align}
	and the standard Newton method is taken into consideration
	\begin{align}
		\label{eq:standardNewton}
		\begin{cases}
			x^{(k+1)} = x^{(k)} + \alpha_k d^{(k)} \\
			d^{(k)}   = - {\left( \nabla^2 f\Big(x^{(k)}\Big) \right)}^{-1} \nabla f\Big(x^{(k)}\Big)
		\end{cases}
	\end{align}
	\begin{enumerate}[(i)]
		\item Compute $3$ steps of the Newton method using $\alpha_k = 1/2$ and $x^{(0)} = -3/2$.
		\item Given $\alpha_k = 1$. Check all possibilities of the starting points,
		      and examine whether the standard Newton method is converged, diverged, or oscillating.
	\end{enumerate}
\end{exampleboxed}
Approach:
\begin{enumerate}[(i)]
	\item First, we shall need to compute
	      the gradient $\nabla f(x)$
	      as follows
	      \begin{align}
		      \label{eq:gradient1}
		      \nabla f (x)  
		      = 
		      \frac{d}{dx}f(x)
		      =
		      \frac{\big(1+x^2\big)'}{2\sqrt{1+x^2}}
		      =
		      \frac{x}{\sqrt{1+x^2}}
	      \end{align}
	      and the second gradient $\nabla^2 f$ is computed as follows
	      \begin{align}
		      \label{eq:gradient2}
		      \nabla^2 f(x) 
		      = 
		      \frac{d^2}{dx^2}f(x)
		      =
		      \frac{d}{dx}\left( \frac{x}{\sqrt{1+x^2}} \right)
		      =
		      \big(1+x^2\big)^{-3/2}
	      \end{align}
	      
	      Next, the direction $d^{(k)}$ for the standard Newton method is ready computed
	      \begin{align}
		      d^{(k)}
		       & = - {\left( \nabla^2 f\Big(x^{(k)}\Big) \right)}^{-1} \nabla f\Big(x^{(k)}\Big) \\
		       & = - {\left( 1+{\Big(x^{(k)}\Big)}^2 \right)}^{3/2}
		      \frac{x^{(k)}}{{\left( 1+{\Big(x^{(k)}\Big)}^2 \right)}^{1/2}}                     \\
		       & = - x^{(k)} \left(1 + {\left(x^{(k)} \right)}^2 \right)
	      \end{align}
	      
	      Then, taking into consideration $\alpha_{k} = 1/2$
	      the next step of the standard Newton method takes the form
	      \begin{align}
		      x^{(k+1)}
		       & = x^{(k)} + \alpha_k \, d^{(k)}                                        \\
		       & = x^{(k)} - \frac{1}{2} x^{(k)} \Bigg(1 + {\Big(x^{(k)}\Big)}^2 \Bigg)
	      \end{align}
	      
	      With the starting point $x^{(0)} = -3/2$ we gradually obtain the next steps
	      \begin{enumerate}
		      \item First step:
		            \begin{align}
			            x^{(1)} 
			            = x^{(0)} - \frac{1}{2} x^{(0)} \Bigg(1 + {\Big(x^{(0)}\Big)}^2 \Bigg)
			            = \dots
		            \end{align}
		            
		      \item Second step:
		            \begin{align}
			            x^{(2)} 
			            = x^{(1)} - \frac{1}{2} x^{(1)} \Bigg(1 + {\Big(x^{(1)}\Big)}^2 \Bigg)
			            = \dots
		            \end{align}
		            
		      \item Third step:
		            \begin{align}
			            x^{(3)} 
			            = x^{(2)} - \frac{1}{2} x^{(2)} \Bigg(1 + {\Big(x^{(2)}\Big)}^2 \Bigg)
			            = \dots
		            \end{align}
	      \end{enumerate}
	      
	\item Examine the convergence
	      \begin{observationboxed}
		      When studying the convergence of the line search in general or 
		      the Newton method in particular,
		      we shall need to know first the minimum $x^*$ of the function $f(x)$,
		      so that we may examine the following expression
		      \begin{align}
			      \label{eq:limitconvergence}
			      \lim_{k \to \infty} \Big| x^{(k+1)} - x^* \Big|.
		      \end{align}
		      Should the scheme converges, the limit results in $0$.
	      \end{observationboxed}
	      By taking into consideration \eqref{eq:gradient1}
	      we may compute the extremum 
	      \begin{align}
		      \nabla f (x)  
		      = 
		      \frac{x}{\sqrt{1+x^2}}
		      = 0
		      \Rightarrow
		      x^* = 0.
	      \end{align}
	      Then, the extremum $x^* = 0$, which is a local minimum
	      by considering the \eqref{eq:gradient2},
	      can be inserted into the 
	      \eqref{eq:limitconvergence}
	      to examine the converged ability
	      \begin{align}
		      \lim_{k \to \infty} \Big| x^{(k+1)} - x^* \Big| \stackrel{!}{=} 0
		       & \Leftrightarrow
		      \lim_{k \to \infty} \Big| x^{(k+1)} - 0 \Big| \stackrel{!}{=} 0                                 \\
		       & \Leftrightarrow
		      \lim_{k \to \infty} \Big| x^{(k+1)} - 0 \Big| \stackrel{!}{=} 0                                 \\
		       & \Leftrightarrow \lim_{k \to \infty} \Big| - { \Big( x^{(k)} \Big)}^3 \Big| \stackrel{!}{=} 0
		      \stackrel{(\star)}{\Rightarrow}
		      \big| x^{(0)} \big|  < 1
	      \end{align}
	      where $(\star)$ is enabled only if the radius of the starting point $x^{(0)}$ is less than $1$.

		  \clearpage
	      Therefore, it leads us to the following conclusion
	      \begin{enumerate}[(i)]
		      \item \( \big| x^{(0)} \big| < 1 \): The method will be converged.
		      \item \( \big| x^{(0)} \big| = 1 \): The method will be oscillating, i.e. by flipping around.
		      \item \( \big| x^{(0)} \big| > 1 \): The method will be diverged.
	      \end{enumerate}
	      
	      Note in passing that we have computed the Newton step 
	      in the context of $\alpha_k = 1$ as follows
	      \begin{align}
		      x^{(k+1)}
		       & = x^{(k)} + \alpha_k \, d^{(k)}                                    \\
		       & = x^{(k)} - 1 \cdot x^{(k)} \Bigg(1 + {\Big(x^{(k)}\Big)}^2 \Bigg) \\
		       & = - \Big( x^{(k)} \Big)^3
	      \end{align}
\end{enumerate}


%------------------------------------------------------------------------------
\clearpage
\section{Demo programming exercise PRU03}
 (White board + Projector)
\begin{remarkboxed}
	Rosenbrock function
	\begin{align}
		f:
		\begin{cases}
			\mathbb{R} \times \mathbb{R} \to \mathbb{R}, \\
			(x,y) \mapsto f(x,y) := a\, (y-x^2)^2 + (1-x)^2,
		\end{cases}
	\end{align}
	where $a=100$,
	is well-known in Optimization. It is often used as a test problem 
	for evaluating algorithms in Optimization.
	This function has a \textbf{global minimum} $x^* = 0$ at the point $(1,1)$.
\end{remarkboxed}


\bibliographystyle{acm}
% \bibliography{literature.bib}
\end{document}