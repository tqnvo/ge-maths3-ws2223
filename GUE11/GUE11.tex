\documentclass[12pt]{article}
\usepackage{anysize}
\usepackage[ngerman,english]{babel}
\marginsize{3.5cm}{2.5cm}{1cm}{2cm}

\usepackage[utf8]{inputenc}
\usepackage[english]{babel}
\usepackage{amsmath}
\usepackage{amsthm}
\usepackage{amsfonts}
\usepackage{scalerel,amssymb}  % for mathbb
%---------------------------------------------------------
\usepackage{authblk}
\usepackage{blindtext}
\usepackage{graphicx}
\usepackage[caption=true]{subfig}
\usepackage[
	colorlinks = true,
	citecolor = red,
	% linkcolor = darkblue, % internal references
	% urlcolor = darkblue,
]{hyperref}
\usepackage{cleveref}
\crefname{figure}{Figure}{Figures}
\crefname{equation}{Eq.}{Eqs.}
%---------------------------------------------------------
% \newcommand{\begin{equation}}{\eqbeg} 
% \newcommand{\end{equation}}{\eqend}
\usepackage{bm}
\usepackage{dfcmd}
% \newcommand{\Bvarepsilon}{\bm\varepsilon} 
% \newcommand{\Bx}{\bm x} 
% \newcommand{\Bu}{\bm u} 
% \newcommand{\Bv}{\bm v} 
% \newcommand{\Bg}{\bm g} 
% \newcommand{\Bb}{\bm b} 
\newcommand{\divv}{\text{divv}}
%---------------------------------------------------------
\usepackage{lineno}
% \linenumbers     

\usepackage{float}
\usepackage{psfrag}

\usepackage{fontawesome}
\usepackage{relsize}

%---------------------------------------------------------
%For \toprule \midrule \bottomrule in table environment
\usepackage{booktabs}

%For theorem into a box
\usepackage{mdframed}
% \newmdtheoremenv{theo}{Theorem}
\newmdtheoremenv{interestingfactboxed}{Interesting fact}
\newmdtheoremenv{observationboxed}{Observation}
\newmdtheoremenv{exampleboxed}{Example}
\newmdtheoremenv{questionboxed}{Question}
\newmdtheoremenv{summaryboxed}{Summary}

%For big integral sign
\usepackage{bigints}

%For [i)] in enumerate
\usepackage{enumerate}

%For cancelto
\usepackage{cancel}
%Page numbering in style 1/3...
\usepackage{lastpage}  
\usepackage{hyperref}
\makeatletter
\renewcommand{\@oddfoot}{\hfil 
% Aachen, November $04^{th}$, 2021 \hspace{300pt} 
Mathe III $\cdot$ GUE11 $\cdot$ WS22/23 
\hspace{250pt} 
\thepage/\pageref{LastPage}\hfil}
\makeatother
%------------------------------------------------------------------------------
\begin{document}
\begin{center}
	\section*{Global exercise - GUE11}
\end{center}
\begin{center}
	Tuan Vo
\end{center}
\begin{center}
	$12^{\text{th}}$ January, 2023
\end{center}
\textbf{\textsc{Summary of content covered:}} Analysis
\begin{itemize}
	\item Review line integral of the 1st kind (scalar field) and 2nd kind (vector field)
	\item Surface integral of the 1st kind (scalar field) and 2nd kind (vector field)
	\item Gauss's theorem and its useful applications
\end{itemize}
%------------------------------------------------------------------------------
\section{Course evaluation}

The evaluation for the course

\begin{center}
	\textbf{22W-11.03530 (L) Mathematische Grundlagen III (CES) (Übung)}
\end{center}

has been opened from now on until \textbf{Friday 20.01.2023 23:59:00}
via QR code 

\inputfig{floats/evaluation_QR}{evaluation_QR}

or via the following link 

\url{https://www.campus.rwth-aachen.de/evasys/online.php?pswd=8J5FHC8JH9}

%------------------------------------------------------------------------------
\clearpage
\section{Summary: Line integral and Surface integral}
\subsection{First kind: Scalar field}
\begin{recallboxed}
	\label{recall:scalar}
	Line integral of a 
	\textbf{scalar field} $\phi: \Omega \to \mathbb{R}$ is defined as follows
	\begin{align}
		\bigintsss_{\Gamma} \phi \, ds 
		:= \bigintsss_{a}^{b} \phi(\gamma(t))  \, \big\| \gamma'(t) \big\| \, dt
	\end{align}
\end{recallboxed}
$\rightarrow$ Mass
\begin{recallboxed}
	\label{recall:scalarsurface}
	Surface integral of a 
	\textbf{scalar field} $\phi: \Omega \to \mathbb{R}$ is defined as follows
	\begin{align}
		\bigintsss_{\Gamma} \phi \, dA 
		:= \bigintsss_{B} \phi(\Phi(p,q)) \, \big\| \Phi_p \times \Phi_q \big\| \, d(p,q)
	\end{align}
\end{recallboxed}
$\rightarrow$ Mass
%------------------------------------------------------------------------------
\subsection{Second kind: Vector field}
\begin{recallboxed}
	\label{recall:vectorsurface}
	Line integral of a \textbf{vector field} $\Bf: \Omega \to \mathbb{R}^{n}$ is defined as follows
	\begin{align}
		\bigintsss_{\Gamma} \Bf \cdot d\Bx 
		:= \bigintsss_{a}^{b}
		\Big\langle \Bf(\gamma(t)), \gamma'(t) \Big\rangle \, dt
	\end{align}
\end{recallboxed}
$\rightarrow$ Work done
\begin{recallboxed}
	\label{recall:vector}
	Surface integral of a \textbf{vector field} $\Bf: \Omega \to \mathbb{R}^{n}$ is defined as follows
	\begin{align}
		\bigintsss_{\Gamma} \langle \Bf, \Bv \rangle \, dA 
		= \bigintsss_{\Gamma} \Bf \cdot d\BA
		:= \bigintsss_{B}
		\Big\langle 
		\phi(\Phi(p,q)), \Phi_p \times \Phi_q
		\Big\rangle
		\, d(p,q)
	\end{align}
\end{recallboxed}
$\rightarrow$
%------------------------------------------------------------------------------
\clearpage
\section{Step-by-step: Surface integral of the 1st kind}
\begin{exampleboxed}
	Examine
	Given surface integral
	\begin{align}
		\bigintsss_{\Omega} 2\sqrt{4 - x^2 - y^2} \, d\sigma
		\label{eq:surf_int}
	\end{align}
	where $\Omega = S \cap Z$ is the portion of the sphere
	\begin{align}
		S = \Big\{ (x,y,z) \in \mathbb{R}^3 :z \geq 0,\, x^2 + y^2 + z^2 = 4 \Big\}
	\end{align}
	inside the cylinder
	\begin{align}
		Z = \Big\{ (x,y,z) \in \mathbb{R}^3 : x^2+y^2 \leq 1 \Big\}
	\end{align}
	\begin{enumerate}
		\item Define spherical coordinates in $\Omega$ (parameterize the surface of interest $\Omega$).
		\item Calculate the surface integral (\ref{eq:surf_int}) using the defined spherical coordinates.
	\end{enumerate}
\end{exampleboxed}
Approach: 
On the surface $\Omega$, we have $r = 2$, so we choose kind of sphere coordinates with:
\begin{align}
	\Phi(\phi,\theta)
	= (2\cdot \cos \phi \cdot \sin \theta, 2\cdot \sin \phi \cdot \sin \theta, 2 \cdot \cos \theta)^T,
	~~ \phi \in [0,2\pi], \theta \in [0,\frac{\pi}{6}]
\end{align}
Hence, we obtain
\begin{align}
	\Bigg\| \frac{\partial \Phi}{\partial \phi}\times\frac{\partial \Phi}{\partial \theta} \Bigg\|
	= 2^2 \cdot |\sin \theta|
\end{align}

Furthermore, on the surface we have
$x^2+y^2 = 4~$
so that
\begin{align}
	2 \sqrt{4-x^2-y^2} = 2\sqrt{z^2} = 2 \cdot z = 4 \cdot \cos \theta
\end{align}
Therefore, we obtain
\begin{align}
	\therefore\quad\boxed{
		\begin{aligned}
			\bigintsss_{\Omega} 2\sqrt{4 - x^2 - y^2} \, d\sigma
			 & =
			\bigintsss\limits_{0}^{2\pi}
			\bigintsss\limits_{0}^{\frac{\pi}{6}} 4 \cos(\theta) \cdot 2^2 \sin \theta d\theta d\phi     \\
			 & = 32 \pi \bigintsss\limits_{0}^{\frac{\pi}{6}} \cos \theta \cdot \sin(\theta) \,  d\theta \\
			 & = 4 \pi
		\end{aligned}
	}
\end{align}

%------------------------------------------------------------------------------
\clearpage
\section{Step-by-step: Surface integral of the 2nd kind}
\begin{exampleboxed}
	Examine the 
	surface integral of a vector field based on 
	the surface
	\begin{align}
		F =
		\left\{
		\alpha\in[0,1],\; \beta \in \Big[ 0,2\pi \Big)
		\,\Bigg|\,
		\Big( e^\alpha, \alpha\cos(\beta), \alpha\sin(\beta) \Big)
		\right\},
	\end{align}
	and the vector field  $\Bf$ given by
	$\Bf(x,y,z) = \Big( x\sqrt{y^2+z^2}, y, z \Big)$.
\end{exampleboxed}
Approach:

\begin{enumerate}[(i)]
	\item The normal vector of the surface is computed as follows
	      \begin{align}
		      \Bn
		       & 
		      = \partial_{\alpha} F \times \partial_{\beta} F
		      = \begin{pmatrix} 
			        e^\alpha \\\cos(\beta)\\\sin(\beta)
		        \end{pmatrix}
		      \times
		      \begin{pmatrix} 
			      0 \\ -\alpha\sin(\beta) \\  \alpha\cos(\beta)
		      \end{pmatrix}                                                          \\
		       & 
		      = \begin{pmatrix} 
			        \alpha\cos^2(\beta) + \alpha\sin^2(\beta) \\ -\alpha\cos(\beta)e^\alpha\\ -\alpha\sin(\beta)e^\alpha
		        \end{pmatrix} \\
		       & 
		      = \begin{pmatrix} 
			        \alpha \\ -\alpha\cos(\beta)e^\alpha\\ -\alpha\sin(\beta)e^\alpha
		        \end{pmatrix},
	      \end{align}
	      \begin{observationboxed}
		      Be aware of the sign switching $(+,-,+)$ when using \textbf{cofactor method} to compute the determinant 
		      \begin{align}
			      \partial_{\alpha} F \times \partial_{\beta} F
			       & =
			      \begin{vmatrix}
				      i        & j                  & k                 \\
				      e^\alpha & \cos(\beta)        & \sin(\beta)       \\
				      0        & -\alpha\sin(\beta) & \alpha\cos(\beta)
			      \end{vmatrix} \\
			       & =
			      i \begin{vmatrix} \cos(\beta) & \sin(\beta) \\ -\alpha\sin(\beta) & \alpha\cos(\beta) \end{vmatrix}
			      - 
			      j \begin{vmatrix} e^\alpha & \sin(\beta) \\ 0 & \alpha\cos(\beta) \end{vmatrix}
			      + 
			      k \begin{vmatrix} e^\alpha & \cos(\beta) \\ 0        & -\alpha\sin(\beta) \end{vmatrix}
		      \end{align}
	      \end{observationboxed}
	      and the norm of the normal vector is computed as follows
	      \begin{align}
		      \| \Bn \|_2
		      = \sqrt{\alpha^2 + \alpha^2\cos^2(\beta)e^{2\alpha} + \alpha^2\sin^2(\beta)e^{2\alpha}}
		      = \alpha \sqrt {1+e^{2\alpha}}.
	      \end{align}
	      which leads to the normalized normal vector of the surface $F$ as follows
	      \begin{align}
		      \boxed{
			      \bm{\nu}
			      = \left(1+e^{2\alpha}\right)^{-1/2}
			      \begin{pmatrix} 1\\-\cos(\beta)e^\alpha\\-\sin(\beta)e^\alpha\end{pmatrix}.
		      }
	      \end{align}
	      
	\item We now compute the surface integral of the second kind as follows
	      \begin{align}
		      \boxed{
			      \begin{aligned}
				      \bigintsss_F
				      \Big\langle \Bf , \bm{\nu} \Big\rangle \,d\sigma
				       & = \bigintsss_0^1 \!\!\! \bigintsss_0^{2\pi}
				      \Big\langle \Bf\big(F(\alpha,\beta)\big), \bm{\nu} \Big\rangle
				      \,d\beta \, d\alpha                                                                                               \\
				       & = \bigintsss_0^1 \!\!\! \bigintsss_0^{2\pi}
				      \begin{pmatrix}
					      e^\alpha\left(\alpha^2\cos^2(\beta)+\alpha^2\sin^2(\beta)\right)^{1/2} \\ 
					      \alpha\cos(\beta)                                                      \\
					      \alpha\sin(\beta)
				      \end{pmatrix}
				      \cdot
				      \begin{pmatrix}
					      \alpha                     \\ 
					      -\alpha\cos(\beta)e^\alpha \\ 
					      -\alpha\sin(\beta)e^\alpha
				      \end{pmatrix}\;  d\beta \,  d\alpha                                                                               \\
				       & = 
				      \bigintsss_0^1 \!\!\! \bigintsss_0^{2\pi}
				      \Big( \alpha^2 e^\alpha - \alpha^2\cos^2(\beta)e^\alpha - \alpha^2\sin^2(\beta)e^\alpha \Big)\; d\beta \, d\alpha \\
				       & = \dots
			      \end{aligned}
		      }
	      \end{align}
\end{enumerate}

%------------------------------------------------------------------------------
\clearpage
\section{Step-by-step with Gauss's theorem example}
\begin{exampleboxed}
	Examine the vector field $\bm{f}$ with
	\begin{align}
		\bm{f}:
		\begin{cases}
			\mathbb{R}^2 \rightarrow \mathbb{R}^2, \\
			(x,y) \mapsto \bm{f}(x,y)
			= \begin{pmatrix} x+y, y^2\end{pmatrix},
		\end{cases}
	\end{align}
	and let $B\subset \mathbb{R}^2$ be a triangle with vertices at $(0,0),(1,0)$ and $(1,2).$
	Verify the Gauss's theorem, by calculating the two integrals
	$\bigintsss_{\partial B} \bm{f}\cdot \bm{\nu} \, ds$ and
	$\bigintsss_B \div \bm{f} \, dA$, and compare them.
\end{exampleboxed}
Approach: 
\begin{observationboxed}[2nd kind line integral]
	We note that the integral
	\begin{align}
		\bigintsss_{\partial B} \bm{f}\cdot \bm{\nu} \, ds
	\end{align}
	is the line integral of the second kind, i.e.
	line integral of a vector field.
\end{observationboxed}
\begin{observationboxed}[1st kind surface integral]
	We note that the integral
	\begin{align}
		\bigintsss_B \div \bm{f} \, dA
	\end{align}
	is the surface integral of the first kind, i.e.
	surface integral of a scalar field. 
	Note in passing that divergence of vector field $ \Bf \in \mathbb{R}^3$ yields a scalar 
	\begin{align}
		\div \Bf = 
		\begin{pmatrix} \partial_x \\ \partial_y \\ \partial_z \end{pmatrix}
		\cdot 
		\begin{pmatrix} f_1 \\ f_2 \\ f_3 \end{pmatrix}
		=
		\partial_x f_1 + \partial_yf_2 + \partial_zf_3.
	\end{align}
	Likewise, divergence of vector field $ \Bf \in \mathbb{R}^2$ yields a scalar
	\begin{align}
		\div \Bf = 
		\begin{pmatrix} \partial_x \\ \partial_y \end{pmatrix}
		\cdot 
		\begin{pmatrix} f_1 \\ f_2 \end{pmatrix}
		=
		\partial_x f_1 + \partial_yf_2.
	\end{align}
\end{observationboxed}

\begin{enumerate}[(i)]
	\item First, we compute the surface integral of the scalar field
	      \begin{align}
		      \therefore\quad
		      \boxed{
			      \begin{aligned}
				      \bigintsss_B \div \bm{f} \, dA
				       & =\bigintsss_0^1 \!\!\! \bigintsss_0^{2x}
				      \div
				      \begin{pmatrix} x+y\\y^2 \end{pmatrix} \, dy \, dx                     \\
				       & =\bigintsss_0^1 \!\!\! \bigintsss_0^{2x} \Big(1+2y\Big) \, dy \, dx \\
				       & =\bigintsss_0^1 \Big( 4x^2 + 2x \Big) \, dx                         \\
				       & = \dots
			      \end{aligned}
		      }
	      \end{align}
	      where we have computed the divergence of vector field $\Bf$ as follows
	      \begin{align}
		      \div \Bf 
		       & = 	
		      \begin{pmatrix} \partial_x \\ \partial_y \end{pmatrix}
		      \cdot 
		      \begin{pmatrix} f_1 \\ f_2 \end{pmatrix}
		      =
		      \begin{pmatrix} \partial_x \\ \partial_y \end{pmatrix}
		      \cdot 
		      \begin{pmatrix} x+y \\ y^2 \end{pmatrix} \\
		       & 
		      = \partial_x f_1 + \partial_yf_2         \\
		       & 
		      = \partial_x (x+y) + \partial_y(y^2)     \\
		       & 
		      = \dots
	      \end{align}
	      
	\item Since the area $B$ is bounded by the three curves, we obtain the following relations
	      \begin{align}
		      \boxed{
			      \begin{aligned}
				      \gamma_1(t)
				       & = \begin{pmatrix}t\\0\end{pmatrix}, \, t\in \left[0,1\right]
				       & \quad \rightarrow
				       & \quad
				      \gamma_1'(t) = \begin{pmatrix} 1\\0 \end{pmatrix}
				       & \quad \rightarrow
				      \quad
				      \Vert \gamma'_1 \Vert_2
				       & = 1                                                                 \\
				      %-------------------------------------------------
				      \gamma_2(t)
				       & = \begin{pmatrix} 1\\ t \end{pmatrix}, \, t\in \left[ 0,2 \right]
				       & \quad \rightarrow
				       & \quad
				      \gamma_2'(t) = \begin{pmatrix} 0\\ 1 \end{pmatrix}
				       & \quad \rightarrow
				      \quad
				      \Vert \gamma'_2 \Vert_2
				       & = 1                                                                 \\
				      %-------------------------------------------------
				      \gamma_3(t)
				       & = \begin{pmatrix}1-t\\2-2t\end{pmatrix}, \, t\in \left[ 0,1 \right]
				       & \quad \rightarrow
				       & \quad
				      \gamma_3'(t) = \begin{pmatrix} -1 \\ -2 \end{pmatrix}
				       & \quad \rightarrow
				      \quad
				      \Vert \gamma'_3 \Vert_2
				       & = \sqrt{5}
			      \end{aligned}
		      }
	      \end{align}
	      \clearpage
	      \begin{observationboxed}
		      Parametrization of a line function may lead to various options. 
		      However, we should be aware of the limit of the parameter variable.
		      For example
		      \begin{align}
			      \gamma_2(t) = 
			      \begin{pmatrix} 1 \\ 0 \end{pmatrix}
			      + t
			      \left(
			      \begin{pmatrix} 1 \\ 2 \end{pmatrix}
			      -
			      \begin{pmatrix} 1 \\ 0 \end{pmatrix}
			      \right)
			      =
			      \begin{pmatrix} 1 \\ 2t \end{pmatrix}
		      \end{align}
		      which leads to new limit applied on $t$ taking $\left[ 0,1 \right]$
		      instead of $\left[ 0,2 \right]$
		      \begin{align}
			      \gamma_2(t)
			       & = \begin{pmatrix} 1\\ 2t \end{pmatrix}, \, t\in \left[ 0,1 \right]
			       & \quad \rightarrow
			       & \quad
			      \gamma_2'(t) = \begin{pmatrix} 0\\ 2 \end{pmatrix}
			       & \quad \rightarrow
			      \quad
			      \Vert \gamma'_2 \Vert_2
			       & = 2
		      \end{align}
	      \end{observationboxed}
	      
	      Likewise, normal vectors take the form
	      \begin{align}
		      \boxed{
			      \bm{\nu}_1 = \begin{pmatrix}0\\-1\end{pmatrix} \text{ on}\ \gamma_1(t),
			      \qquad
			      \bm{\nu}_2 = \begin{pmatrix}1\\0\end{pmatrix} \text{ on}\ \gamma_2(t),
			      \qquad
			      \bm{\nu}_3 = \begin{pmatrix}-2\\1\end{pmatrix}\frac{1}{\sqrt{5}} \text{ on}\ \gamma_3(t),
		      }
	      \end{align}
	      which have been computed based on the following relations
	      \begin{align}
		      \bm{\nu}_{1}\cdot\gamma_{1}'(t) \stackrel{!}{=} 0 \\
		      \bm{\nu}_{2}\cdot\gamma_{2}'(t) \stackrel{!}{=} 0 \\
		      \bm{\nu}_{3}\cdot\gamma_{3}'(t) \stackrel{!}{=} 0 
	      \end{align}
	      
	      Now, we compute the line integral of the second kine
	      $\bigintsss_{\partial B} \bm{f}\cdot \bm{\nu} \, ds$
	      \begin{align}
		      \bigintsss_{\partial B} \bm{f}\cdot\bm{\nu} \, ds
		       & =
		      \bigintsss_{\partial B = \gamma_1(t) + \gamma_2(t) + \gamma_3(t)} \bm{f}\cdot\bm{\nu} \, ds
		      \\
		       & =
		      \bigintsss_{\gamma_1(t)} \bm{f}\cdot\bm{\nu}_1 \, ds +
		      \bigintsss_{\gamma_2(t)} \bm{f}\cdot\bm{\nu}_2 \, ds +
		      \bigintsss_{\gamma_3(t)} \bm{f}\cdot\bm{\nu}_3 \, ds                                  \\
		       & =
		      \bigintsss_{\gamma_1(t)} \bm{f}(\gamma_1'(t)) \cdot\bm{\nu}_1 \, ds +
		      \bigintsss_{\gamma_2(t)} \bm{f}\cdot\bm{\nu}_2 \, ds +
		      \bigintsss_{\gamma_3(t)} \bm{f}\cdot\bm{\nu}_3 \, ds
		      \\
		       & =\bigintsss_0^1
		      \begin{pmatrix}t+0 \\0^2 \end{pmatrix} \cdot \begin{pmatrix}0 \\-1 \end{pmatrix} \, dt
		      + \bigintsss_0^2
		      \begin{pmatrix}1+t \\t^2 \end{pmatrix} \cdot \begin{pmatrix}1 \\0 \end{pmatrix} \, dt \\
		       & \quad\quad\quad\quad\quad\quad\quad
		      + \bigintsss_0^1
		      \begin{pmatrix}1-t+2-2t \\ 4-8t+4t^2\end{pmatrix} \cdot \begin{pmatrix}-2\\1\end{pmatrix}
		      \frac{\sqrt{5}}{\sqrt{5}} \, dt                                                       \\
		       & =\bigintsss_0^2 (1+t) \, dt
		      + \bigintsss_0^1 (4t^2-2t-2) \, dt
		      = \frac{7}{3}.
	      \end{align}
\end{enumerate}


Therefore, we have verified the Gauss's theorem
\begin{align}
	\therefore\quad
	\boxed{
		\bigintsss_{\partial B} \bm{f} \cdot \bm{\nu}
		= \bigintsss_B \div \bm{f} \, d\lambda_2.
	}
\end{align}


\bibliographystyle{acm}
% \bibliography{literature.bib}
\end{document}