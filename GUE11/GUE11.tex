\documentclass[12pt]{article}
\usepackage{anysize}
\usepackage[ngerman,english]{babel}
\marginsize{3.5cm}{2.5cm}{1cm}{2cm}

\usepackage[utf8]{inputenc}
\usepackage[english]{babel}
\usepackage{amsmath,amsfonts,amssymb,amsthm}
% \usepackage{mathtools}
\usepackage{scalerel}  % for mathbb
%---------------------------------------------------------
\usepackage{authblk}
\usepackage{blindtext}
\usepackage{graphicx}
\usepackage[caption=true]{subfig}
\usepackage[
	colorlinks = true,
	citecolor = red,
	% linkcolor = darkblue, % internal references
	% urlcolor = darkblue,
]{hyperref}
\usepackage{cleveref}
\crefname{figure}{Figure}{Figures}
\crefname{equation}{Eq.}{Eqs.}
%---------------------------------------------------------
% \newcommand{\begin{equation}}{\eqbeg} 
% \newcommand{\end{equation}}{\eqend}
\usepackage{bm}
\usepackage{dfcmd}
% \newcommand{\Bvarepsilon}{\bm\varepsilon} 
% \newcommand{\Bx}{\bm x} 
% \newcommand{\Bu}{\bm u} 
% \newcommand{\Bv}{\bm v} 
% \newcommand{\Bg}{\bm g} 
% \newcommand{\Bb}{\bm b} 
\newcommand{\divv}{\text{divv}}
%---------------------------------------------------------
\usepackage{lineno}
% \linenumbers     

\usepackage{float}
\usepackage{psfrag}

\usepackage{fontawesome}
\usepackage{relsize}

%---------------------------------------------------------
%For \toprule \midrule \bottomrule in table environment
\usepackage{booktabs}

%For theorem into a box
\usepackage{mdframed}
% \newmdtheoremenv{theo}{Theorem}
\newmdtheoremenv{interestingfactboxed}{Interesting fact}
\newmdtheoremenv{observationboxed}{Observation}
\newmdtheoremenv{remarkboxed}{Remark}
\newmdtheoremenv{recallboxed}{Recall}
\newmdtheoremenv{exampleboxed}{Example}
\newmdtheoremenv{questionboxed}{Question}
\newmdtheoremenv{summaryboxed}{Summary}

%For big integral sign
\usepackage{bigints}

%For [i)] in enumerate
\usepackage{enumerate}

%For cancelto
\usepackage{cancel}

%For color
\usepackage[dvipsnames]{xcolor}

%Norm package
\usepackage{commath}
% \usepackage[sc,osf]{mathpazo}
% \let\oldnorm\norm   % <-- Store original \norm as \oldnorm
% \let\norm\undefined % <-- "Undefine" \norm
% \DeclarePairedDelimiter\norm{\lVert}{\rVert}
%Page numbering in style 1/3...
\usepackage{lastpage}  
\usepackage{hyperref}
\makeatletter
\renewcommand{\@oddfoot}{\hfil 
% Aachen, November $04^{th}$, 2021 \hspace{300pt} 
Mathe III $\cdot$ GUE11 $\cdot$ WS22/23 
\hspace{250pt} 
\thepage/\pageref{LastPage}\hfil}
\makeatother
%------------------------------------------------------------------------------
\begin{document}
\begin{center}
	\section*{Global exercise - GUE11}
\end{center}
\begin{center}
	Tuan Vo
\end{center}
\begin{center}
	$12^{\text{th}}$ January, 2023
\end{center}
Content covered:
\begin{itemize}
	\item[\checkmark] Analysis: Surface integral of the first kind (scalar field)
	\item[\checkmark] Analysis: Surface integral of the second kind (vector field)
	\item[\checkmark] Analysis: Gauss's theorem
\end{itemize}
%------------------------------------------------------------------------------
\section{Analysis: Surface integral}
\begin{exampleboxed}
	Examine the lateral surface of the \textbf{Frustum} given as follows
	\begin{align*}
		K =
		\left\{
		\bm{x} \in \mathbb{R}^3,\,
		0 < r \leq R
		\, \Bigg| \,
		0 \leq x_3 < H,\,
		0 \leq x_1^2 + x_2^2 <
		\left(R - \frac{R-r}{H} x_3\right)^2
		\right\}.
	\end{align*}
	where $H$ is the height, and $r$ is the radius.
\end{exampleboxed}
Approach:
By using the cylinder coordination 
$\bm{x} = (r \cos (\phi), r \sin(\phi),z)^T$
we obtain the parametrization of the lateral surface of the frustum,
i.e. truncated cone
\begin{align*}
	\bm{\gamma}: 
	\begin{cases}
		(0,2\pi)\times (0,H) \to \mathbb{R}^3, \\
		(\phi,z) \mapsto \bm{\gamma}(\phi,z) :=
		\begin{pmatrix} \left(R - \frac{R-r}{H}z\right) \cos(\phi) \\
			\left(R - \frac{R-r}{H}z\right) \sin(\phi) \\
			z
		\end{pmatrix}
	\end{cases}
\end{align*}
which leads to the following expression
\begin{align*}
	\boxed{
		\begin{aligned}
			\partial_{\phi} \bm{\gamma}(\phi,z)\times \partial_z \bm{\gamma}(\phi,z)
			 & = \begin{pmatrix} -\left(R - \frac{R-r}{H}z\right) \sin(\phi) \\
				     \left(R - \frac{R-r}{H}z\right) \cos(\phi)  \\
				     0
			     \end{pmatrix}
			\times
			\begin{pmatrix}
				-\frac{R-r}{H}\cos(\phi) \\-\frac{R-r}{H} \sin(\phi)\\1
			\end{pmatrix}             \\
			 & = \begin{pmatrix}
				     \left(R- \frac{R-r}{H}z\right)\cos(\phi) \\
				     \left(R- \frac{R-r}{H}z\right)\sin(\phi) \\
				     \left(R- \frac{R-r}{H}z\right)\frac{R-r}{H}
			     \end{pmatrix}
		\end{aligned}
	}
\end{align*}
where the normalization yields
\begin{align*}
	\boxed{
		\Big\|
		\partial_{\phi} \bm{\gamma} (\phi,z) \times \partial_z \bm{\gamma} (\phi,z)
		\Big\|
		= \left(R - \frac{R-r}{H}z\right) \sqrt{1 + \frac{(R-r)^2}{H^2}}.
	}
\end{align*}

After that, the lateral surface of the truncated cone 
is computed as follows
\begin{align*}
	\mathcal{A}
	=
	\bigintsss_{0}^H \bigintsss_0^{2 \pi}
	\Big\| 
	\partial_{\phi} \bm{\gamma} \times \partial_z \bm{\gamma}
	\Big\|
	\, d\phi \, dz                            
	= 2 \pi \sqrt{1 + \frac{(R-r)^2}{H^2}}
	\bigintsss_{0}^H \left(R - \frac{R-r}{H}z\right) \, dt 
	= \dots
	% = \pi \sqrt{1 + \frac{(R-r)^2}{H^2}}\frac{H}{R-r} \left(R^2-r^2\right).
\end{align*}

Therefore, we obtain
\begin{align*}
	\therefore\quad\boxed{
		\mathcal{A} =
		\bigintsss_{0}^H \bigintsss_0^{2 \pi}
		||\partial_{\phi} \bm{\gamma} \times \partial_z \bm{\gamma} || \, d\phi \, dz
		= \dots.
	}
\end{align*}
%------------------------------------------------------------------------------
\clearpage
\section{Numerics: Demo programming exercise 3 PRU(02)}
\inputfig{floats/pe03_QR_2}{pe03_QR_2}
\inputfig{floats/pe03_QR_4}{pe03_QR_4}
\begin{observationboxed}
	For the sake of rapid checks and coding prototype
	we may first start with a matrix sized $4\times 4$
	for the case $C_{ij}=\sin(ij)$ as given in the programming exercise.
\end{observationboxed}
\inputfig{floats/pe03_QR_2_plots}{pe03_QR_2_plots}

\bibliographystyle{acm}
% \bibliography{literature.bib}
\end{document}