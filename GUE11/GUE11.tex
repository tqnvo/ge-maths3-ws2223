\documentclass[12pt]{article}
\usepackage{anysize}
\usepackage[ngerman,english]{babel}
\marginsize{3.5cm}{2.5cm}{1cm}{2cm}

\usepackage[utf8]{inputenc}
\usepackage[english]{babel}
\usepackage{amsmath,amsfonts,amssymb,amsthm}
% \usepackage{mathtools}
\usepackage{scalerel}  % for mathbb
%---------------------------------------------------------
\usepackage{authblk}
\usepackage{blindtext}
\usepackage{graphicx}
\usepackage[caption=true]{subfig}
\usepackage[
	colorlinks = true,
	citecolor = red,
	% linkcolor = darkblue, % internal references
	% urlcolor = darkblue,
]{hyperref}
\usepackage{cleveref}
\crefname{figure}{Figure}{Figures}
\crefname{equation}{Eq.}{Eqs.}
%---------------------------------------------------------
% \newcommand{\begin{equation}}{\eqbeg} 
% \newcommand{\end{equation}}{\eqend}
\usepackage{bm}
\usepackage{dfcmd}
% \newcommand{\Bvarepsilon}{\bm\varepsilon} 
% \newcommand{\Bx}{\bm x} 
% \newcommand{\Bu}{\bm u} 
% \newcommand{\Bv}{\bm v} 
% \newcommand{\Bg}{\bm g} 
% \newcommand{\Bb}{\bm b} 
\newcommand{\divv}{\text{divv}}
%---------------------------------------------------------
\usepackage{lineno}
% \linenumbers     

\usepackage{float}
\usepackage{psfrag}

\usepackage{fontawesome}
\usepackage{relsize}

%---------------------------------------------------------
%For \toprule \midrule \bottomrule in table environment
\usepackage{booktabs}

%For theorem into a box
\usepackage{mdframed}
% \newmdtheoremenv{theo}{Theorem}
\newmdtheoremenv{interestingfactboxed}{Interesting fact}
\newmdtheoremenv{observationboxed}{Observation}
\newmdtheoremenv{remarkboxed}{Remark}
\newmdtheoremenv{recallboxed}{Recall}
\newmdtheoremenv{exampleboxed}{Example}
\newmdtheoremenv{questionboxed}{Question}
\newmdtheoremenv{summaryboxed}{Summary}

%For big integral sign
\usepackage{bigints}

%For [i)] in enumerate
\usepackage{enumerate}

%For cancelto
\usepackage{cancel}

%For color
\usepackage[dvipsnames]{xcolor}

%Norm package
\usepackage{commath}
% \usepackage[sc,osf]{mathpazo}
% \let\oldnorm\norm   % <-- Store original \norm as \oldnorm
% \let\norm\undefined % <-- "Undefine" \norm
% \DeclarePairedDelimiter\norm{\lVert}{\rVert}
%Page numbering in style 1/3...
\usepackage{lastpage}  
\usepackage{hyperref}
\makeatletter
\renewcommand{\@oddfoot}{\hfil 
% Aachen, November $04^{th}$, 2021 \hspace{300pt} 
Mathe III $\cdot$ GUE11 $\cdot$ WS22/23 
\hspace{250pt} 
\thepage/\pageref{LastPage}\hfil}
\makeatother
%------------------------------------------------------------------------------
\begin{document}
\begin{center}
	\section*{Global exercise - GUE11}
\end{center}
\begin{center}
	Tuan Vo
\end{center}
\begin{center}
	$12^{\text{th}}$ January, 2023
\end{center}
Summary of content covered: Analysis
\begin{itemize}
	\item Review line integral of the 1st kind (scalar field) and 2nd kind (vector field)
	\item Surface integral of the 1st kind (scalar field) and 2nd kind (vector field)
	\item Gauss's theorem and its useful applications
\end{itemize}
%------------------------------------------------------------------------------
\section{Course evaluation}

%------------------------------------------------------------------------------
\clearpage
\section{Analysis: Line integral and Surface integral}
\subsection{First kind: Scalar field}
\begin{recallboxed}
	\label{recall:scalar}
	Line integral of a 
	\textbf{scalar field} $\phi: \Omega \to \mathbb{R}$ is defined as follows
	\begin{align}
		\bigintsss_{\Gamma} \phi \, ds 
		:= \bigintsss_{a}^{b} \phi(\gamma(t)) \big\| \gamma'(t) \big\| \, dt
	\end{align}
\end{recallboxed}
\begin{recallboxed}
	\label{recall:scalarsurface}
	Surface integral of a 
	\textbf{scalar field} $\phi: \Omega \to \mathbb{R}$ is defined as follows
	\begin{align}
		\bigintsss_{\Gamma} \phi \, ds 
		:= \bigintsss_{a}^{b} \phi(\gamma(t)) \big\| \gamma'(t) \big\| \, dt
	\end{align}
\end{recallboxed}
%------------------------------------------------------------------------------
\subsection{Second kind: Vector field}
\begin{recallboxed}
	\label{recall:vectorsurface}
	Line integral of a \textbf{vector field} $\Bf: \Omega \to \mathbb{R}^{n}$ is defined as follows
	\begin{align}
		\bigintsss_{\Gamma} \Bf \cdot d\Bx 
		:= \bigintsss_{a}^{b} \big\langle \Bf(\gamma(t)), \gamma'(t) \big\rangle \, dt
	\end{align}
\end{recallboxed}
\begin{recallboxed}
	\label{recall:vector}
	Line integral of a \textbf{vector field} $\Bf: \Omega \to \mathbb{R}^{n}$ is defined as follows
	\begin{align}
		\bigintsss_{\Gamma} \Bf \cdot d\Bx 
		:= \bigintsss_{a}^{b} \big\langle \Bf(\gamma(t)), \gamma'(t) \big\rangle \, dt
	\end{align}
\end{recallboxed}
%------------------------------------------------------------------------------
\clearpage
\section{Analysis: Surface integral of the first kind}
\begin{exampleboxed}
	Examine
\end{exampleboxed}
Approach:

%------------------------------------------------------------------------------
\clearpage
\section{Analysis: Surface integral of the second kind}
\begin{exampleboxed}
	Examine
\end{exampleboxed}
Approach:

%------------------------------------------------------------------------------
\clearpage
\section{Analysis: Gauss theorem}
\begin{exampleboxed}
	Examine
\end{exampleboxed}
Approach:

\bibliographystyle{acm}
% \bibliography{literature.bib}
\end{document}